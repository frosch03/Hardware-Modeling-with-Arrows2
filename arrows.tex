%\documentclass[11pt,final,a4paper,leqno]{article}
\documentclass[11pt,final,a4paper]{article}
\usepackage[ngerman,english]{babel}
%\usepackage{ucs}
\usepackage[T1]{fontenc}
\usepackage{color}
\usepackage{graphicx}
\usepackage{subfigure}
\usepackage{listings}
\usepackage{fancyvrb}
\usepackage{textcomp}
\usepackage{setspace}
%\usepackage{MnSymbol}
\usepackage{amsmath}

\lstnewenvironment{haskell}[1][]{
  \lstset{
    fancyvrb=true, columns=flexible, language=haskell,
    captionpos=b,
    basicstyle=\ttfamily\small\setstretch{1},
    commentstyle=\color{ggray}\slshape,
    upquote=true,
    escapechar={@},
    emph={True,False}, emphstyle=\color{green},
    literate=*{...}{{\textcolor{ggray}{...}}}{3}%
             {\{}{{\textcolor{blue}{\{}}}{1}%
             {\}}{{\textcolor{blue}{\}}}}{1}%
             {->}{{$\rightarrow$}}{4},
    stringstyle=\color{red}, showstringspaces=false,
    keywordstyle=\color{blue},#1
}}{}


% Commands
\newcommand{\hs}[1]{%
  \lstinline[
    fancyvrb=true, columns=flexible, language=haskell,
    captionpos=b,
    basicstyle=\ttfamily\setstretch{1},
    commentstyle=\color{ggray}\slshape,
    upquote=true,
    escapechar={@},
    emph={True,False}, emphstyle=\color{green},
    literate=*{...}{{\textcolor{ggray}{...}}}{3}%
             {\{}{{\textcolor{blue}{\{}}}{1}%
             {\}}{{\textcolor{blue}{\}}}}{1}%
             {->}{{$\rightarrow$ }}{2},
    stringstyle=\color{red}, showstringspaces=false,
    keywordstyle=\color{blue}]!#1!%
}
\newcommand{\grafik}[4][0.9]{%        % Bild einfügen, [Skalierung], {Dateiname ohne Endung}, {Beschriftung}, {label zum Referenzieren}
    \begin{figure}[ht]%               % before: htbp 
        \begin{center}
            \includegraphics[width=#1\columnwidth]{Images/#2}
            \caption{\label{#4} #3}
            
        \end{center}
    \end{figure}
}
\newcommand{\xfig}[4][0.9] {%      %xfig figure einfügen, [Skalierung], {Dateiname ohne Endung}, {Beschriftung}, {label}
    \begin{figure}[ht]
        \begin{center}
            \graphicspath{{./}{Images/}}
            \scalebox{#1}{%
                \input{Images/#2}
            }
            \caption{\label{#4} #3}
        \end{center}
    \end{figure}
}
\newcommand{\reFLect}{\textit{re\kern-0.07em F\kern-0.07emL\kern-0.29em\raisebox{0.56ex}{ect}}}
\newcommand{\boxit}[1]{\mbox{{\it #1}}}

% Colors
\definecolor{gray}{gray}{0.5}
\definecolor{ggray}{gray}{0.7}
\definecolor{green}{rgb}{0,0.5,0}
\definecolor{textgray}{gray}{0.25}
\definecolor{backgray}{gray}{0.95}


\title{Hardware Design with Functional Datastructures: Functions, Arrows, Tupels, Vectors, and Inspection} 

\begin{document}

\section{Introduction}

\section{Hardware Modeling in the Functional Programming Paradigm: Recent Approaches}
\label{recent_approaches}
% TODO : Is stated in intro? 
As stated in the introduction, modeling hardware in functional languages is old news, there have been various approaches in the
past. The beginning dates back to the early 80s where a group around Mary Sheeran came up with a functional HDL (Hardware
Description Language) muFP\cite{sheeran:muFP}. In the same decade John O'Donnell presented his functional HDL
HDRE\cite{hydra:old,donnell}. These two HDL's gave birth to numerous later approaches that are base more or less on them.

One of Sheeran's students, Coen Claessen, later came up with a monadic solution called Lava\cite{claessen:hardware}. Ingo Sanders
et al introduced an approach, ForSyDe \cite{forsyde:phd,forsyde:ieee} that is based on meta programming techniques. Lava, ForSyDe
and also Hydra %TODO : Hydra reference in literature
represent HDL's embedded around the functional programming language Haskell; MyHDL \cite{myhdl} is embedded in the functional
flavored interpreter language Python. There is HML\cite{hml}, a simple HDL which is embedded inside ML, another functional
programming language.

Than there are functional languages that are exclusively designed to describe hardware, like the language SAFL (=Statically
Allocated Functional Language)\cite{sharp,sharp:flash,sharp:codesign}. This one compiles direct into a netlist; it does not
support dynamic allocated storage like heaps or stacks since these kind of software features do not map well on silicone.

Another candidate for the functional meta programming languages section is the language \reFLect \cite{reflect}. It's one that is
developed at Intel and tailored for hardware design and theorem proving.

There are good reasons why meta programming beneficial for the use as HDLs. Meta programming allows a program to have a
representation of itself, usually by providing a data structure of the abstract syntax tree. With this tool at hand the developer
is able to \emph{compute} parts of their program rather than \emph{write} them. This leads to a direct and natural implementation
of program transformations.

With cryptol\cite{cryptol:programming} another class of functional HDLs enters the scene. While all the other HDL's are general
purpose, cryptol is designed to model cryptographic hardware. A sub-language of cryptol is designed by Galois, explicit to
generate FPGA implementations \cite{cryptol:fpga}. Cryptol has been adopted by an American agency and isn't public developed
anymore.

\section{Arrows for Hardware development}
% TODO : Cite: Generalizing Monads to arrows
Arrows are lately developed concept in functional programming. As a generalization of Monads, Arrows fit the task of hardware
development better. Arrows enable a data flow view onto problems. This fits the hardware developers view which has always been a
data flow view, as in circuit board layouts or in state flow diagrams. The arrow \emph{proc}-notation also transports the data
flow view into the textual source. %TODO: cite ross patterson
Its easy to generate netlists out of arrows, which basically boils down to a pretty print function. This also holds for the
generation of typical HDL code like VHDL. They can be used to simulate behavior or proof properties of the circuit. This section
gives a brief overview of the \hs{Grid}-Arrow that can be used to generate VHDL code or to simulate a circuit.

\subsection{Grid-Arrow}
\label{subsect:grid}
The \hs{Grid}-Arrow is a combination of an executable arrow and a descriptive hardware component part. In a first approach one can
imagine the descriptive part to live inside the arrow. But that does not work out, so they are stored equal to each other inside a
tuple. The arrow resides in the first, the graph occupies the second part of the tuple.
\begin{haskell}
  newtype Grid a b c = GR (a b c, CircuitDescriptor)
\end{haskell} 
Alongside with the arrow execution, the generated Netlist is stored inside graph structure called a \hs{CircuitDescriptor}.  The
name Grid is analog chosen to the snap circuit toys from childhood, where one also could combine hardware elements arbitrary. The
Grid type is a surface where hardware elements can be combined with each other to compose new structures and analyze them. It is
not only the surface to combine hardware components, it's also recursive. So it is used as hardware component itself which means a
\hs{Grid} is an arrow on its own.


\subsection{Circuit-Descriptor}
The descriptive part of the arrow is called a \hs{CircuitDescriptor}, and describes a circuit in detail. The combination sequence
of the hardware parts, the pin layout, their cabling structure and also recursively the subcomponents are represented inside the
\hs{CircuitDescriptor}. A circuit with such a descriptor alongside has enough details to be expressed as a Netlist.

\par
There are three distinct forms of circuits:
\begin{itemize}
\item \emph{Combinatorial:} are the parts of a circuit that are describable without a clock or buffered data. Boolean logic gates
  or deterministic finite automata are in this category.
\item \emph{Register:} A register is used to describe the storage of data on the one hand and is used to clock computations on the
  other hand. The implementation of a register highly depends on the target platform. There is no need to know the details at
  this point.
\item \emph{Loop:} Hardware components that are referring onto them self are handled in the looping section.
\end{itemize}

\begin{haskell}
  data CircuitDescriptor
    = MkCombinatorial
      { nodeDesc :: NodeDescriptor
      , nodes    :: [CircuitDescriptor]
      , edges    :: [Edge]
      , cycles   :: Tick
      , space    :: Area
      }
    | MkRegister
      { nodeDesc :: NodeDescriptor
      , bit      :: Int
      }
    | MkLoop
      { nodeDesc :: NodeDescriptor
      , nodes    :: [CircuitDescriptor]
      , edges    :: [Edge]
      , cycles   :: Tick
      , space    :: Area
      }
    | NoDescriptor
    deriving (Eq)
  type Netlist = CircuitDescriptor
\end{haskell} 

The \hs{MkCombinatorial} and the \hs{MkLoop} parts reveal a functional graph structure that is build from two lists. One
containing the edges and another one containing nodes. Additionally there are equal information for every circuit. They are stored
inside the \hs{NodeDescriptor} with the graph. At last, information about the hardware's performance (space and
time) is stored alongside with the graph.
\begin{haskell}
  data NodeDescriptor
    = MkNode
      { label   :: String 
      , nodeId  :: ID
      , sinks   :: Pins
      , sources :: Pins
      }
    deriving (Eq)
\end{haskell} 
% Each node itself is a \hs{CircuitDescriptor} and contains a list of \hs{CircuitDescriptor}s, so each node could be described
% recursively, which suits hardware very well.

To avoid corrupted data types, smart constructors are used. They ensure correct formed graphs by checking the incoming data
against given constraints and then generate only valid graphs. For example the combinatorial section only accepts non looping
connections.

\par
A wire is a connection between \hs{Pins} of one or
two hardware components. The output is called the source and the input is called the sink of a component. 


\subsection{Lists through Tuples}
The Haskell arrow interface is build from tuples. Circuits come often with a list like interface. There are operators like in
binary logic that perfectly match the tuple structure, but in most cases the list interface is needed. Circuits ofte come with a
list like interface and therefore we modell this throug tuples. The process of restructuring the output of one function into the
input to another function is called retuple'ing. A function that does such tuple'ing is \hs{aFloatR}. This one takes an argument
of the structure \((x,(y,z))\) to a different structure \((y,(x,z))\).

\begin{haskell}
aFloatR :: (Arrow a) => Grid a (my, (b, rst)) (b, (my, rst))
aFloatR           
  =   aDistr      -- aDistr : is that really aDistr?
  >>> first aSnd
\end{haskell} 

There are two arrows involved here, \hs{aDistr} and \hs{aSnd}. With \hs{aDistr} the first element is distributed over the second
one and with \hs{first aSnd} the first occurrence is dropped. The \hs{aFloatR} arrow moves the first element one position to the
right. Of curse this arrow assumes a structure where the first element of a tuple is a value, the second is another tuple: \((v_1,
(v_2, (v_3, (v_4, v_5, (\dots)))))\)

\par
A common task over the list like tuples is to apply an arrow to the front and than concentrate the remaining calculation on the
inner structure. For this purpose the operator \hs{>:>} helps to clarify the code that describes such computations.
\begin{haskell}
  aA >:> aB = aA >>> (second aB)
\end{haskell} 

The operator \hs{>:>} is a special case of the \hs{>>>} operator. The first supplied arrow is executed over the whole input, while
the second one is applied to only the second part. Such an operator comes handy as long tuple structures are used and are
processed from the left.

\par
While working with tuple lists, it is needed to reorder the parenthesis around the data so that selector functions like \hs{first}
process the right data. This is done by the two arrows \hs{a\_xXX2XXs} and \hs{a\_XXx2xXX} that exploit the associative law of
arrows. The names mirror the changing paranthesis structure, upper case letters show the grouped position.
\begin{align*}
  \text{a\_xXX2XXx} &: (a, (b, c)) \rightarrow ((a, b), c) \\
  \text{a\_XXx2xXX} &: ((a, b), c) \rightarrow (a, (b, c))
\end{align*}



\section{Classical Arrows Hardware Development}
Within this section we present a cycle redundancy check algorithm, implemented with arrows. Later on it's shown how netlists are
generated. The section concludes by highlighting some of the problems with this approach.

\subsection{CRC - Implementation}
A CRC - cycle redundancy check - is a method to calculate checksums of data. The calculation of a CRC is based upon a polynomial
division, where the reminder is the CRC. Our approach calculates the CRC with a functional galois type shift register.

\par
The shift register is modeled from a list and the positions of where to feedback. The list is simulated by tuples and therefore
the feedback value is feed back into the list by the \hs{aFloatR} arrow.

\par
Different feedback positions are captured in multiple arrows called \hs{aFeedback$_n$}. 

\begin{haskell}
aFeedback5
  =   aFloatR
  >:> aFloatR
  >:> aFloatR
  >:> aFloatR
  >:> aFlip
\end{haskell} 

% TODO : aCRC8
The principle behind every CRC is captured in the arrow \hs{crc\_checksum\_8} 
\begin{haskell}
crc_checksum_8 crc_polynom polySkip padding start rest --------------------REVIEW
  =   (padding &&& aId)
  >>> aFeedback8

  >>> (start &&& rest)
  >>> first (crc_polynom)

  >>> step
  >>> step
  >>> step
  >>> step
  >>> step
  >>> step

  >>> aFlip
  >>> polySkip
  >>> crc_polynom

  where step =   a_xXX2XXx
             >>> first 
                 (   aFlip
                 >>> polySkip
                 >>> crc_polynom
                 )   
\end{haskell} 

The CRC input data are alignet and the padding bits are floated to the right end of the tuple list.
\[((p_0, p_1), (b_0, (b_1, (b_2, (b_3, b_4))))) \rightarrow (b_0, (b_1, (b_2, (b_3, (b_4, (p_0, p_1))))))\]

After that, the arrow is an abstract representation of the CRC calculation. By parameterising over \hs{crc_checksum_8}, every
concrete CRC algorithm can be constructed. Due to Haskells inability to dynamicaly change the count of certain expressions in the
function body, our example is limited to 8Bit.


\subsection{Example}
A CRC-Code is equipped with a polynomial that especially fits the needs of the domain. This example shows the CRC-Code for USB. With the polynomial:
\[
  \text{USB} : x^5 + x^2 + 1 
\]
 
The polynomial describes which positions are feed back and are \hs{xor}'ed with the bit that has traveled through the shift
register. For the USB CRC $\text{bit}_0$ and $\text{bit}_2$ are \hs{xor}'ed with $\text{bit}_5$.

\xfig[.5]{CRC-USB}{USB CRC}{fig:usb-crc}

To summarize this for the USB CRC, the highest exponent is $5$. Therefore:
% TODO : Namings aFeedback ... feedback usw. ... 
\begin{itemize}
  \item the \hs{polySkip} needs to be \hs{aFeedback5}
  \item there are $5$ \hs{padding} bits \hs{aConst (False, (False, (False, (False, False))))} 
  \item the split between \hs{start} and \hs{rest} is after bit $5$
\end{itemize}


% \begin{haskell}
% crc_checksum_ccit_8 
%   :: Arrow a => Grid a 
%     (Bool, (Bool, (Bool, (Bool, (Bool, (Bool, (Bool, Bool))))))) 
%     (Bool, (Bool, (Bool, Bool)))
% crc_checksum_ccit_8
%   = crc_checksum_8   
%       crc_polynom_ccit
%       aFeedback4
%       (aConst (False, (False, (False, False))))
%       (second.second.second.second $ aFst)
%       (aSnd >>> aSnd >>> aSnd >>> aSnd >>> aSnd)
% \end{haskell} % $ to fix that syntax highlighting problem 


\subsection{Building Netlist-Graphs}
With a netlist being an universal description of the hardware circuits it is our goal to convert the hardware describing arrow
into one. 

\par
The execution of the described functions produce a \hs{Grid} arrow as result. As stated in Section \ref{subsect:grid} a Grid
consists of the arrow representing the behavior and the graph representing the structure that causes this behavior. Every
component comes with a trivial graph containing only one node. That node describes the component in detail. 

\par
The hardware components are combined with each other by arrow operators to generate new components with different
behavior. Alongside with the combinations of the hardware a new describing graph is generated. This graph describes now, from what
components the new one is build up. And it also describes how the single parts are interconnected.

\par
To interconnect two circuits into a new one, a function \hs{splice} is used. This function is abstract in the sense that it can
splice two circuits together in different ways. They correspond to the arrow operations. For example graphs that are combined by
\hs{(>>>)} are spliced with the \hs{seqRewire} method. The function connects the sources of circuit 1 with the sinks of circuit 2
one by one. After that step all the remaining sinks of the two circuits are extracted into one list and all the remaining sources
are extracted into another list. These two lists become the new sinks and sources of the generated component. To generate a proper
graph of circuits one has to make sure, that all the identifiers inside the components are again unique. 

\xfig[.5]{combine-components}{combine two components}{fig:combine}

\par
It turns out, that this graph in fact is a form of a netlist. Once the graph is created, it is straight forward to translate it
into another netlist format (e.g. VHDL).

\subsection{Drawbacks}
This approach has problems, that are visible in the CRC example. The input data of these hardware components must fit into
tuples. Logic gates fit well into this scheme, but for hardware components with more than two pins data structures like lists are
needed. Lists can be simulated with tuples as it is done in the above example. But the simulation comes at a cost. A big part of
the code only restructures the tuples. This process becomes tedious, confusing and error prone. At least if bigger real world
hardware components are be modeled. Think about 32 or even 64-Bit ALU's.

\par
The tedious tuple code could be avoided by using Patterson's proc-notation for arrows\cite{PatersonNewNotation}.  The
proc-notation abstracts the tuples to named variables. The hardware can be described more ``visually'' in the source. But this is
just syntactic sugar, there are still tuples in use. The following code shows how the USB CRC is modeled with the proc-notation:
\begin{haskell}
------------------------------------------------ REVIEW wrong function
crc_polynom_ccit 
 = proc (b4, (b3, (b2, (b1, b0)))) -> do
     b4' <- aId  -< (b3)
     b3' <- aId  -< (b2)
     b2' <- aXor -< (b4, b1)
     b1' <- aXor -< (b4, b0)
     returnA     -< (b4', (b3', (b2', b1')))
\end{haskell} 

\par
The second problem affects the resulting auto generated arrow code. Patterson's notation heavily uses the \hs{arr}
function. Because it is possible to ``lift'' any function passed to \hs{arr} into an arrow, it is impossible to keep track of that
outcoming function type. The type information of the passed function is needed to generate a proper component. This makes it
possible to generate component arrows ad-hoc. Nonetheless a proper name to that function is missing in every case.


\section{Coping with the Tupel-Problem - ``Vector-Arrows''}
Arrows match binary hardware pretty good. In this section we show howto overcome the problems with $n$-ary ($n > 2$) hardware
components.

% TODO : wo passt dieser satz? 
% So arrows are an adequate abstract form to describe hardware, they are modular and they come with an enhanced type system that
% helps to reduce errors. The part which does not match the hardware perfect is the way the in- and output pins are structured,
% the tuples.

\par
The reason for the arrows preference of binary circuits is it's tuple structure. Also the tuples allow the in- and output to be of
any kind. In the digital world, hardware transmits nearly always Boolean values and also integers are transmitted via bundled
Boolean values. And even the developers of Sensors try to digitize pretty early in their design process. \cite{MakinwaSmartSensor}

\par
Therefore the type \hs{a b c} is far too general for a logic gate: the type of the in going data should match the type of the out
coming values. Secondly, a set of wires is not a tuple and for instance, the tuple \hs{(a,(b,(c,d)))} represents the same set of
wires as the tuple \hs{(a,((b,c),d))}. The number of possible -- different with respect to syntax and type -- representations of a
set of wires exponentially explodes with a growing number of wires that we are trying to model.

\par
The type that would be ideal for hardware development is something that is static in their types but is dynamic in the length.
Arrows in contrast are dynamic in the type but are static in their amount of the values. So arrows are opposing the hardware's
nature here. It is possible to transform arrows so that they match hardware more naturally, without loosing the beneficial
properties for hardware design.

\par
A vector fulfills the desired type needs of the hardware behavior. The vector $Vec_n^b$ is one of $n$ values over the type $b$. In
the following we sketch different ways to express such ``Vector-Arrows'' and present different approaches.

\begin{haskell}
------------------------------------------------REVIEW only one example ? 
class VArrow a where
  arr     :: ((@$Vec_n^{b}$@)-> (@$Vec_m^{b}$@)) -> a (@$Vec_n^{b}$@) (@$Vec_m^{b}$@)
  firsts  :: a (@$Vec_n^{b}$@) (@$Vec_m^{b}$@) -> a (@$Vec_{(n:+k)}^{b}$@) (@$Vec_{(m:+k)}^{b}$@)
  seconds :: a (@$Vec_n^{b}$@) (@$Vec_m^{b}$@) -> a (@$Vec_{(k:+n)}^{b}$@) (@$Vec_{(k:+m)}^{b}$@)
  (>^>)   :: a (@$Vec_n^{b}$@) (@$Vec_m^{b}$@) -> a (@$Vec_m^{b}$@) (@$Vec_k^{b}$@) -> a (@$Vec_n^{b}$@) (@$Vec_k^{b}$@)
  (&^&)   :: a (@$Vec_n^{b}$@) (@$Vec_m^{b}$@) -> a (@$Vec_n^{b}$@) (@$Vec_k^{b}$@) -> a (@$Vec_n^{b}$@) (@$Vec_{(m:+k)}^{b}$@)
  (*^*)   :: a (@$Vec_n^{b}$@) (@$Vec_m^{b}$@) -> a (@$Vec_k^{b}$@) (@$Vec_j^{b}$@) -> a (@$Vec_{(n:+k)}^{b}$@) (@$Vec_{(m:+j)}^{b}$@)
\end{haskell}


\subsection{Horn-like-clause Solution}
As vectors are not a purely functional data structure and lists come pretty close to them the first approach would be to constrain
a list data type in length. It is necessary to introduce a counter, into the data type. We choose type level natural numbers to
restrict the list in length.
% TODO, cite richtig platzieren
\cite{kiselyov:strong_collections}
\begin{haskell}
data Vec n a where
  T    :: Vec 'VZero a                           -- (@$Vec_0^a$@)
  (:.) :: a -> Vec n a -> Vec ('VSucc n) a -- (@$a \rightarrow Vec_n^a \rightarrow Vec_{n+1}^a$@)
\end{haskell} 

The type level numbers are introduced as peano numbers. This is similar to the mechanism shown by Kiselyov et al
\cite{kiselyov:strong_collections}.

\par
\hs{arr} takes a Haskell function into an arrow that expects a vector of $n$ $b$'s ($Vec_n^b$) into a new vector of $b$'s with
length $m$ ($Vec_m^b$). The sequential concatenation of arrows is done by the \hs{(>^>)} operator. \hs{firsts} translates an arrow
into another one, that takes it's input from the front element of a tuple. \hs{second} does the same thing, but with the second
element of that tuple. So these two functions are uniquely based upon the tuple. With vector arrows there are no tuples, and
therefore there is no way to differentiate a first value from a second one at the moment. 

\par
As Haskell is statically typed, everything in the data type is declared. All variables that hold type level numbers must be
defined in the data type context. If such type level numbers are added, a type class \hs{VAdd} must be listed in the type
context. Working with this contexts reminds of Prolog's Horn Clauses programming style. 
% The arrow class for vector arrows looks like the following:

% \begin{haskell}
%   class (Category a) => VArrow a where
%     arr    :: (VNat n)  
%            => (Vec n b -> Vec n b) -> a (Vec n b) (Vec n b)


%     init   :: ( VNat n, VNat m
%               , VAdd n m nm
%               , VEq (VSucc n) nm VTrue
%               )   
%            => a (Vec n b) (Vec m b) -> (Vec nm b) (Vec nm b) -> a (Vec n b) (Vec n b)

%     tail   :: ( VNat n, VNat m
%               , VAdd n m nm
%               , VEq nm (VSucc m) VTrue
%               )   
%            => a (Vec nm b) (Vec nm b) -> a (Vec m b) (Vec m b)

%     tail'  :: ( VNat m ) 
%            => a (Vec (VSucc m) b) (Vec (VSucc m) b) -> a (Vec m b) (Vec m b)

%     head   :: ( VNat n, VNat m
%               , VAdd n m nm
%               , VEq nm (VSucc m) VTrue
%               )   
%            => a (Vec nm b) (Vec nm b) -> a (Vec n b) (Vec n b)

%     last   :: ( VNat n, VNat m
%               , VAdd n m nm
%               , VEq (VSucc n) nm VTrue
%               )   
%            => a (Vec nm b) (Vec nm b) -> a (Vec m b) (Vec m b)


%     (***)  :: (VNat n, VNat m, VAdd n m nm) 
%            => a (Vec n b) (Vec n b) -> a (Vec m b) (Vec m b) -> a (Vec nm b) (Vec nm b)

%     (&&&)  :: (VNat n, VAdd n n nn) 
%            => a (Vec n b) (Vec n b) -> a (Vec n b) (Vec n b) -> a (Vec n b) (Vec nn b)
% \end{haskell}

\par
The classic arrow class gives a default definition for the \hs{(***)} operator, where \hs{(***)} is defined via \hs{first} and
\hs{second}. With this arrow class there is nothing like a \hs{second} arrow and therefore it's not possible to define
\hs{(***)} in the standard way. The problem rises with an inability to define a correct split function.

\par
It is possible to divide the list just at the ``right spot'', once an arrow is known. \hs{first f} splits the input
into one with exact the length that \hs{f} expects as input, and a rest vector. So with \hs{aXor} being an arrow that takes two
Booleans to one, \hs{first} applied to \hs{aXor} must be an arrow from at least two Booleans to at least one.
\begin{haskell}
  --      a (Vec 2 Bool) (Vec 1 Bool)
  aXor :: a (Vec (VSucc (VSucc VZero)) Bool) (Vec (VSucc VZero) Bool)

  --              a (Vec 2+k Bool) (Vec 1+k Bool)
  first (aXor) :: (VNat k, VAdd (VSucc (VSucc VZero)) k nk, VAdd (VSucc VZero) k mk)
               => a (Vec ((VSucc (VSucc VZero)) + k) Bool) (Vec ((VSucc VZero) + k) Bool
\end{haskell} 
This unsugared syntax is actually a major drawback. It is not possible to write just $2+k$ instead one needs to add \hs{VAdd
  (VSucc (VSucc VZero)) k nk} as context to the data type. This might only be tedious for easy arrows, it becomes an major error
source for larger ones.


\subsection{Type-Family Solutions}
The next approach is to replace the Horn like clauses by actual calculations. Haskell has lately come up with a type family
extension that comes in handy for this task.

\par
A type family in a way enables to calculate the type constructor of a data type. A feature the horn-clause solution lacks. The
length of the vector is stored in a peano number. As peano numbers are expressed as a bunch of type constructors, type families
are used to calculate the constructor amount. The Haskell-Wiki describes this vividly: A type family is to a data type, what the
method of a type class is to a function.

\begin{haskell}
type family (n::VNat)   :+ (m::VNat) :: VNat
type instance VZero     :+ m     = m 
type instance VSucc n   :+ m     = VSucc (n :+ m)
\end{haskell}

The first Line defines the type family, with an operator \hs{:+} that operates on peano numbers.  With the addition type family
the arrow could be defined more similar to the original type class.
\begin{haskell}
class Category a => VArrow a where
  arr     :: (Vec n b -> Vec m b)
          -> a (Vec n b)         (Vec m b)

  firsts  :: a (Vec n b)         (Vec m b)
          -> a (Vec (n :+ k) b)  (Vec (m :+ k) b)

  seconds :: ((m :+ k) ~ (k :+ m))
          => a (Vec n b)         (Vec m b)
          -> a (Vec (k :+ n) b)  (Vec (k :+ m) b)

  (***)   :: a (Vec n  b)        (Vec m  b)
          -> a (Vec n' b)        (Vec m' b)
          -> a (Vec (n :+ n') b) (Vec (m :+ m') b)

  (&&&)   :: a (Vec n b)         (Vec m  b)
          -> a (Vec n b)         (Vec m' b)
          -> a (Vec n b)         (Vec (m :+ m') b)
\end{haskell}

\par 
The problem with this approach is the compiler. It is not possible for GHC to deduce arithmetics in the type context. This is also
valid for simpler calculations down to even the equality of $n+m = m+n$. It's possible to supply the compiler with these
information by hand. But this are only hints into a direction. At this point it gets tedious to supply hints for easy deductions
and gets unpractical for complicated arithmetic's.


\subsection{Dependently-Typed Solution}
A dependently typed programming language is a language, that mixes expressions with types. It therefore allows the calculate of
types within that same language. 

\par
In summary the previous solutions all try to fix a unitary problem. And this problem depends on the fact that vectors aren't
purely functional. A vectors length must show up in it's data type so that a function can match against it.

\par
The transfer of tuple arrows to vector arrows changes the inner data structure of arrows. Therefore the projection maps from the
inner data type must translate to the vector version. \hs{first} and \hs{second} exactly match the tuple projection maps in
Haskell, namely \hs{fst} and \hs{snd}.

\par 
With vectors the \hs{first} statement is straight forward defined. The \hs{second} expression is the head-scratcher.  The
implementation of \hs{first} constraints it's inputs to only those, that are equal or larger than the ones \hs{f} expects. The
\hs{second} case adds the right alignment on top. This can be achieved by calculating the size of the sub vector, ignoring him and
applying the function to the reminder. It can also be done via reversing the vector multiple times and than applying the function
to the front of the vector.

\par
While \hs{aSwitch} is of type $Vec_n^{Int}$, the expression \hs{seconds aSwitch} must be of type $Vec_{k+n}^{Int}, k \ge 0$. This
is hard to express in a functional data type. In a dependent language this is easily expressible by the data type.

\par
There are multiple languages with dependent types out there. The one with a pretty good connection to Haskell is Agda. The
following describes the arrow class in Agda with dependent types.


\begin{haskell}
record VArrow (_~~>_ : Set -> Set → Set) : Set₁ where 
  field
    arr     : forall {B n m}     -> (Vec B n -> Vec B m) -> (Vec B n) ~~> (Vec B m)
    firsts  : forall {B n m k}   -> Vec B n ~~> Vec B m -> Vec B (n + k) ~~> Vec B (m + k)    
    seconds : forall {B n m k}   -> Vec B n ~~> Vec B m -> Vec B (n + k) ~~> Vec B (m + k) 
    _>>>_   : forall {B n m k}   -> Vec B n ~~> Vec B m -> Vec B m ~~> Vec B k -> Vec B n ~~> Vec B k
    _***_   : forall {B n m k j} -> Vec B n ~~> Vec B m -> Vec B k ~~> Vec B j -> Vec B (n + k) ~~> Vec B (m + j)
    _&&&_   : forall {B n m k}   -> Vec B n ~~> Vec B m -> Vec B n ~~> Vec B k -> Vec B n ~~> Vec B (m + k) 
  infixr 2 _>>>_
  infixr 2 _***_
  infixr 2 _&&&_
\end{haskell}

\begin{itemize}
\item 
  \begin{itemize}
  \item tuple arrows => kartesische produkt kategorien
  \item vector arrows => produkt familien kategorien 
  \end{itemize}

\item problems with agda / dependent typed solutions 
  \begin{itemize}
  \item ebenfalls last lässt sich nicht sauber definieren / Gründe??? 
  \end{itemize}
\end{itemize}

\subsection{Vector-Arrows are not Arrows}
-  :+: and :-:
- \ldots

Conclusion: That is dependent typing and Haskell is not (yet) a dependently typed function programming language. And all in all it
seems that the Vec-Type together with the Arrows-Instances is too much dependent typing for Haskell


\section{Copeing with \hs{arr}-Problem}

\subsection{ForSyDe: Using the AST of the arr argument}
In order to transform a usual Haskell function into an hardware block, ForSyDe stores the AST of that function which is used later
to synthesis VHDL.  \

\subsection{Show-Type-Class (or Typeable)}


\subsection{Generalized Arrows -- avoiding arr}




\bibliographystyle{plain}
\bibliography{Bibliography}
\end{document}
