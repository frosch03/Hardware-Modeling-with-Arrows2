\documentclass[9pt,final,a4paper,leqno]{article}
\usepackage[ngerman,english]{babel}
\usepackage{ucs}
\usepackage[T1]{fontenc}
\usepackage{color}
\usepackage{graphicx}
\usepackage{listings}
\usepackage{MnSymbol}
\usepackage[fleqn]{amsmath} % flush eqations to the left
\usepackage{amssymb}
\usepackage{mathtools}

%\usepackage[pdftex]{color}
% Haskell Code
\usepackage{listings}
\usepackage{fancyvrb}
\usepackage{textcomp}
\usepackage{setspace}

\lstnewenvironment{haskell}[1][]{
  \lstset{
    fancyvrb=true, columns=flexible, language=haskell,
    captionpos=b,
    basicstyle=\ttfamily\small\setstretch{1},
    commentstyle=\color{ggray}\slshape,
    upquote=true,
    emph={True,False}, emphstyle=\color{green},
    literate=*{>>>}{{\textcolor{ggray}{>>>}}}{3}%
             {...}{{\textcolor{ggray}{...}}}{3}%
             {\{}{{\textcolor{blue}{\{}}}{1}%
             {\}}{{\textcolor{blue}{\}}}}{1}%
             %{=>}{{$\Rightarrow$ }}{2}%
             {->}{{$\rightarrow$ }}{2},
    stringstyle=\color{red}, showstringspaces=false,
    keywordstyle=\color{blue},#1
}}{}

\newcommand{\hs}[1]{%
  \lstinline[
    fancyvrb=true, columns=flexible, language=haskell,
    captionpos=b,
    basicstyle=\ttfamily\setstretch{1},
    commentstyle=\color{ggray}\slshape,
    upquote=true,
    emph={True,False}, emphstyle=\color{green},
    literate=*{>>>}{{\textcolor{ggray}{>>>}}}{3}%
             {...}{{\textcolor{ggray}{...}}}{3}%
             {\{}{{\textcolor{blue}{\{}}}{1}%
             {\}}{{\textcolor{blue}{\}}}}{1}%
             %{=>}{{$\Rightarrow$ }}{2}%
             {->}{{$\rightarrow$ }}{2},
    stringstyle=\color{red}, showstringspaces=false,
    keywordstyle=\color{blue}]!#1!%
}

\definecolor{gray}{gray}{0.5}
\definecolor{ggray}{gray}{0.7}
\definecolor{green}{rgb}{0,0.5,0}



\lstnewenvironment{haskell}{%
  \lstset{}% 
  \csname lst@SetFirstLabel\endcsname
}
{%
  \csname lst@SaveFirstLabel\endcsname
}

\lstset {
  basicstyle=\small\ttfamily,
  flexiblecolumns = false,
  basewidth = {0.5em, 0.45em},
  literate = {+}{{$+$}}1 {/}{{$/$}}1 {*}{{$*$}}1 {=}{{$=$}}1
             {>}{{$>$}}1 {<}{{$<$}}1 {\\}{{$\lambda$}}1
             {\\\\}{{\char`\\\char`\\}}1
             {->}{{$\rightarrow$}}2 {>=}{{$\geq$}}2 {<-}{{$\leftarrow$}}2
             {<=}{{$\leq$}}2 {=>}{{$\Rightarrow$}}2 
             {\ .}{{$\circ$}}2 {\ .\ }{{$\circ$}}2
             {>>}{{>>}}2 {>>=}{{>>=}}2
             {|}{{$\mid$}}1
             {<<<}{{$\lll$}}2 {>>>}{{$\ggg$}}2 {-<}{{$\leftY$}}1
}
%\lstset{                                % Definition der Sprache, f?r lstListings 
%        language=Haskell,               % Name der Sprache
%        breaklines=true,                % Sollen zu lange Zeilen umgebrochen werden?
%        captionpos=b,                   % Wo soll die Beschriftung stehen? (b - bottom, t - top)
%        tabsize=2,                      % Wie viele Zeichen springt ein Tab?
%        basicstyle=\small,              % Welche Schriftgr??e soll verwendet werden?
%        morekeywords={},    		% W?rter die als Schl?sselw?rter fett gedruckt werden sollen
%        deletekeywords={} 		% W?rter die nicht als Schl?sselw?rter fett gedruckt werden sollen
%}

\title{Hardware Modeling with Functional Datastructures: Functions, Arrows, Tupels, Vectors, and Inspection} 

\definecolor{textgray}{gray}{0.25}
\definecolor{backgray}{gray}{0.95}

\newcommand{\grafik}[4][0.9]{%        % Bild einfügen, [Skalierung], {Dateiname ohne Endung}, {Beschriftung}, {label zum Referenzieren}
    \begin{figure}[ht]%               % before: htbp 
        \begin{center}
            \includegraphics[width=#1\columnwidth]{Images/#2}
            \caption{\label{#4} #3}
            
        \end{center}
    \end{figure}
}

\newcommand{\xfig}[4][0.9] {%      %xfig figure einfügen, [Skalierung], {Dateiname ohne Endung}, {Beschriftung}, {label}
    \begin{figure}[ht]
        \begin{center}
            \graphicspath{{./}{Images/}}
            \scalebox{#1}{%
                \input{Images/#2}
            }
            \caption{\label{#4} #3}
        \end{center}
    \end{figure}
}

\newcommand{\reFLect}{\textit{re\kern-0.07em F\kern-0.07emL\kern-0.29em\raisebox{0.56ex}{ect}}}

\newcommand{\boxit}[1]{\mbox{{\it #1}}}

\newcommand{\hs}[1]{\mbox{\lstinline[basicstyle=\color{textgray}]!#1!}}
\newcommand{\hsLine}[1]{$$\mbox{\tt{#1}}$$}


\begin{document}

\section{Introduction}
\label{introduction}
With the rise of rapid prototyping and development methodologies in the hardware community, hardware has started to get softer. FPGA's are a
major contributor for this trend. It is only natural to take benefits out of the existing software engineering concepts. This can be seen in
hardware description languages like VHDL or Verilog. Both have taken their syntax from common languages like C and ADA\@. Those two are
typical imperative languages, but there are arguments that declarative languages fit more natural to the problem of hardware modeling.
Functional programming languages like \emph{Haskell} are declarative.

\begin{itemize}
  \item \emph{Referential transference}: While imperative languages force the programmer to prescribe the control flow a declarative
    language is stateless. Stateless means, that the programmer does not describe the control flow, but the compiler is able to decide the
    order of execution. Therefore concurrency isn't something that needs to be simulated (like it's done in System-C) but fits natural in
    the language and the developed code.
  \item \emph{Composition}: Functional programs are build out of the composition of small pieces. Again this attributes to hardware design
    where complex systems are also build out of the composition of smaller pieces.
  \item \emph{Type systems}: Functional languages like Haskell come with an expressive type system which helps to detect errors early and
    also helps the developer to a cleaner design.
  \item \emph{Higher-order functions}: Typical functional design techniques like higher order functions can be used to abstract common
    circuit structures, so that the developers task is to instantiate them in order to get circuits.
  \item \emph{Lazy evaluation}: With the help of lazy evaluating languages it is easy to express infinite data structures, which also maps
    natural to stream processing hardware specifications. 
\end{itemize}

This properties lead to some interesting development possibilities which are not new in the functional programming world, but are rather
unusual if not impossible in languages like VHDL\@. With the use of higher order functions developers are able to transform circuits just to
their needed behavior. The clean detachment of the languages semantics from the circuits specification is the key that leads to an extreme
flexible design. Last but not least the clear expressiveness of functional languages makes it easy to \emph{1:} reason, \emph{2:} realize
and \emph{3:} analyze functional programs. % TODO do i need to cite sheeran here?

\par
In fact, these are old news, functional programming pioneers pointed to the use of functional languages as perfect hardware design languages
long before. \cite{sheeran:perfect_match} %O'Donnell and all the others\ldots

\par
From the authors point of view it is not only natural to describe hardware with the help of functional programming languages, but also it is
the logical step.

\section{Hardware Modeling in Haskell: Recent Approaches}
\label{recent_approaches}
As stated in the introduction, modeling hardware in functional languages is old news, there have been various approaches in the past. The
beginning dates back into the early 80s where Mary Sheeran came up with a functional HDL muFP\cite{sheeran:muFP}. In the same decade John
O'Donnell presented his functional HDL HDRE\cite{hydra:old,donnell}. These two HDL's gave birth to numerous later approaches that are base
more or less on them. 

\par
One of Sheeran's students, Coen Claessen, came later up with a monadic solution called Lava\cite{claessen:hardware}. Ingo Sanders et al
introduced an approach, ForSyDe \cite{forsyde:phd,forsyde:ieee} that is based on meta programming techniques. Lava, ForSyDe and also Mydra %TODO : Hydra reference in literature
represent HDL's embedded into the functional programming language Haskell; MyHDL \cite{myhdl} is embedded in the functional flavored
interpreting language Python. There is HML\cite{hml}, a simple HDL which is embedded inside ML, another functional programming language. 

\par
Than there are functional languages that are exclusively designed to describe hardware, like the language SAFL (=Statically Allocated
Functional Language)\cite{sharp,sharp:flash,sharp:codesign}. This one compiles direct into a netlists; it does not support dynamic allocated
storage like heaps or stacks since these kind of software features do not map well on silicone. 

\par
Another candidate for the functional metaprogramming languages section is the language \reFLect \cite{reflect}. This one is developed at
Intel and tailored for hardware design and theorem proving. 

\par
There are good reasons why metaprogramming (also called \emph{reflexivity}) are beneficial for the use as HDLs. Metaprogramming allows a
program to have a representation of itself, usually by providing data structure of the abstract syntax tree. With this tool at hand the
developer is able to \emph{compute} parts of their program rather than \emph{write}them. This leads to a direct and natural implementation
of program transformations. Haskell also allows to metaprogramming through a library called ``Template Haskell''\cite{haskell:template}. 

\par
With cryptol\cite{cryptol:programming} another class of functional HDLs enters the scene. While all the other HDL's are general purpose,
cryptol is designed to model cryptographic hardware. A sublanguage of cryptol is designed by Galois, explicit to generate FPGA
implementations \cite{cryptol:fpga}. Cryptol has lately been adopted by an American agency and isn't public developed anymore.


\section{Why Arrows?}
In this section we give a short introduction to the concept of arrows; we compare them very brief to monads which are better known in the
functional community. We the show why arrows are a good way to describe hardware. 

\subsection{What are arrows}
The concept of a function is a pillar in functional programming and can't be compared to functions in imperative/procedural programming
languages. In languages of the latter kind, functions are seen as structuring elements that stuff imperative expressions together while also
make them available by a specific identifier, the function name. A function in the functional sense is oriented along the mathematical
definition of a function which is defined as the projection of input parameters to output values. If taken serious, this definition leads
to only deterministic behaving functions.  

\par
So the question is, how to express side effects (state changes like IO) with the use of deterministic functions. This dilemma can be solved
by making the state changes deterministic, which is done via a \emph{world parameter}. A function that takes such a world parameter than
is able to react on the world's state, can apply changes and could deliver it in the return value. With this trick it is possible to express
IO-, random or any side effecting function in a functional language. This world parameter could also be used to gain the control flow back,
such that a specific chain of functions, is executed in the given order. Because it is tedious to keep track of this world parameter, it is
captured inside a specific data structure, which then is called a Monad. With monads the developer is able to determine a specific control
flow. On one side this is better than no control flow but on the other side it is to specific to design parallel computations which are
usual found in hardware designs. 

\par
At this point, the concept of arrows comes to play, which represents the generalization of monads \cite{Hughes98generalisingmonads}. While
the monad is the abstract interface to the control flow of a function, the more general arrow states an abstract interface to all features
of a function.


\subsection{Arrows in the Hardware Design Process}
The similarity of functions on the one hand and hardware blocks on the other hand is well known and demonstrated in the functional
programming literature for the last decades \cite{sheeran:perfect_match,donnell}. While these similarities between those two are impressive,
there are also differences that need to be handled. Every approach that models hardware via functional functions has to face the fact, that
functional functions are extreme powerful. Functional concepts like higher-order functions, curried functions, etc\. doesn't map to silicon at
all. 

\par
We conclude that Haskell functions correlate to logic gates on the one hand and offer some incompatible features on the other hand. To
describe hardware components with functional languages, only a very specific subset of the features of a Haskell function can be mapped to
circuit design. Every functional solution tackles theses differences by some means or other. This gives rise to such a manifold solution
space as described in section~\ref{recent_approaches}. The approaches cope this problem via monads, avoidance of functional features or via
meta programming techniques. 

\par
As monads give the developer a handle to control the execution flow of functional programs, arrows offer a handle not only to the control
flow of functional functions, but to many more features of functions. In example, it is possible to specify the execution order but it is
not mandatory to do so. Such a behavior where parallelism is implicit, naturally fits to the behavior of silicone where parallelism also is
implicit. Another example is the ad-hoc creation of functions, so called lambdas in Haskell. The arrow interface provides a function that
can ``lift'' any other function into an arrow and so this mechanism can be used to specify lambda expressions inside arrows. As said
earlier, arrows provide handles to different building blocks of functional functions. Ad hoc creation of hardware blocks is something that
isn't possible and therefor is not used to model hardware. %% TODO : Kommt der unterschied zur vermeidung functionaler features hier raus?

\begin{itemize}
\item Arrows compromise deep and shallow embedding, and seem to pick the benfits of both.a
\item Arrows resemble premonoidal effect categories and\ldots   are a common way for category theorists to model the
semantics of programming languages in a maximal generic way. In this view, arrows are executable maximum generic
semantic specifications. With this genericity, one can hope to be able to do many things with such a specification:
Execution, Synthesis, Proofing, (dependend on the concrete instance of the arrow)
\item Interface (i.\,e. \hs{***}, \hs{&&&}, \hs{>>>}) is natural for combining hardware blocks
\end{itemize}

\par
Haskells type of a function that takes input of type $b$ to output of type $c$ is denoted as $b \rightarrow c$, with the $(\rightarrow)$
being a \emph{higher-order-type}. This higher order type takes two type arguments (namely $b$ and $c$) and yields a type as result (namely
the function type $(\rightarrow)\ b\ c$). $(\rightarrow)$ is just the prefix notation of the infix type operator $\rightarrow$. To take
control over the features of a function, the function type operator $\rightarrow$ needs to be replaced by something more general; a higher
order type variable $a$. This leads to the prefix written type:

\begin{center}
\begin{minipage}{.2\textwidth}
\begin{haskell}
a b c
\end{haskell}
\end{minipage} \end{center}

Because this is to general to be of any use, Haskells \emph{type classes} are used to specify the properties of that type variable $a$. This
type class is called \hs{Arrow}. 

\begin{haskell}
class Arrow a b c where 
  arr   :: (b -> c) -> a b c
  (>>>)  :: a b c    -> a c d   -> a b d
  (***) :: a b c    -> a b' c' -> a (b, b') (c, c')
  (&&&) :: a b c    -> a b  c' -> a  b      (c, c')
\end{haskell}

In this type class a function \hs{arr} to lift ordinary functions into an arrow, the bind operator \hs{(>>>)}, as well as the combination
operator \hs{(***)} are defined. \hs{(&&&)} is just a special case of \hs{(***)}. 

\par
With the bind operator \hs{(>>>)} two arrows are combined into a new one and the specific arrows are executed in sequence from left to
right. This corresponds directly to sequencing circuit, where one gate computes first with it's outputs solder to the next gate. 
\par 
To model parallel computation the \hs{(***)}-operator is used. It take two different data flows ($b$ and $b'$), passes them into the first
and the second arrow. The results ($c$ and $c'$) are than combined and yielded as overall result. The \hs{(&&&)}-operator takes only one
input and provides it to both inner arrows.

\xfig[.6]{ArrowStarStarStar}{Schematic representation of ***}{figure:ststst}
\xfig[.6]{ArrowAndAndAnd}{Schematic representation of (\&\&\&)}{figure:ananan}

\par
Interestingly, this structure is at the same time the implementation of a lately developed concept in category theory, called Freyd~categories 
which are essentially premonoidal effect categories\cite{Heunen06arrows, Hughes98generalisingmonads,PatersonRA}. The term Freyd
category marks essentially a functor with another category in its domain. The important part is, that it specifies a category of
computations and therefore a mathematical definition for category theorists to model the semantics of programming languages. To have such a
direct connection to category theory helps to underly our approach with mathematical proof in the future. This also shows that the arrow
approach is a generic semantic for describing hardware systems. Also such a generic system not only can be used to model the hardware, but
also to synthesis, code generation, simulation or property proofing. 

\par
And there are more possibilities while using this approach. Another interesting arrow typeclass is the \hs{ArrowChoice} class. With a choice
arrow a model of different execution paths, depending on input parameters, could be expressed. This behavior is translated into Haskell's
\hs{Either} type and so it the result of type \hs{Either}. 

\begin{haskell}[]
class ArrowChoice a where
  (+++) :: a b c -> a b' c' -> Either (b b') (c c')
  (|||) :: a b c -> a b' c  -> Either (b b')  c 
\end{haskell}

Here again are two incarnations of that choice-operator. There is the \hs{(|||)}-operator, which combines two arrows with different input
types to an arrow of the same output type. Than there is the more abstract version, the \hs{(+++)}-operator which not only accepts two
different input types, but also returns two different output types.  

\xfig[.8]{ArrowPlPlPl}{Schematic representation of \hs{+++}}{figure:plplpl}
\xfig[.8]{ArrowPiPiPi}{Schematic representation of \hs{|||}}{figure:pipipi}

\par
For Hardware development interesting is also the \hs{ArrowLoop} typeclass that is able to express recursive circuits where the output value
is feed back into the arrow..
\begin{haskell}[]
class ArrowLoop a where
  loop :: a (b, d) (c, d) -> a b c
\end{haskell}

This one is essential for hardware development because there are considerable few circuits which go without recursion. 

\xfig[.8]{ArrowLp}{Schematic representation of loop}{figure:loop}


\section{First Approach: Classical Arrows}
Within this section, we present a basic cycle redundancy check implementation with the design method of arrows. The example shows a typical
way to express hardware with the functional method and also the flexibility of a functional approach is demonstrated. This is done by
implementing multiple CRC's with differing polynomials. How to generate netlists out of the functional data structure is shown later on.
After that, the problems with classical arrows are highlighted and solutions are discussed.


\subsection{Implementation}
The calculation of a specific CRC to some input data is based upon a mathematical method called a polynomial division. The reminder of that
devision is the CRC Checksum. The original data padded by the CRC results in 0 if that devision is done again. 

\par
First of all a few operators are shown, that help to build the hardware later. There is the function \hs{mvRight} 
\begin{haskell}
mvRight :: (Arrow a) => Grid a (my, (b, rst)) (b, (my, rst))
mvRight 
  =   aDistr
  >>> first aSnd
\end{haskell} 


\par 
There are lots of CRC-Codes out there and every one is equipped with a polynomial that especially fits the needs of its domain. In this
examples the simple CRC-Codes for USB and CCIT are shown. Their polynomials are:
\begin{align*}
  \text{USB} &: x^5 + x^2 + 1 \\
  \text{CCIT}&: x^4 + x   + 1 
\end{align*}



\begin{itemize}
  \item 
    \begin{itemize}
      \item USB  - Polynom
      \item CCIT - Polynom
    \end{itemize}

  \item 
    \begin{itemize}
      \item Operator >:>
      \item Function mvRight
      \item Function toInnerN
    \end{itemize}

  \item
    \begin{itemize}
      \item crc\_checksum\_8
      \item crc\_checksum\_usb\_8
      \item crc\_checksum\_ccit\_8
    \end{itemize}
    
\end{itemize}





%%%%%%%%%%%%%%%%%%%%
%{-# LANGUAGE Arrows,
%             NoMonomorphismRestriction,
%             RebindableSyntax #-}
%module CRC 
%where
%
%import Prelude
%
%import Control.Category ((>>>))
%
%import Circuit
%import Circuit.Arrow
%import Circuit.Defaults
%
%infixr 0 >:>
%
%(>:>) aA aB = aA >>> (second aB)
%
%mvRight :: (Arrow a) => Grid a (my, (b, rst)) (b, (my, rst))
%mvRight
%    =   aDistr
%    >>> first aSnd
%
%
%
%crc_polynom_ccit ::  Arrow a => Grid a (Bool, (Bool, (Bool, (Bool, Bool))))  (Bool, (Bool, (Bool, Bool)))
%crc_polynom_ccit 
%    =   mvRight >:> mvRight >:>
%        (   aDistr 
%        >>> (aXor *** aXor)
%        )
%
%-- crc_polynom_ccit 
%--  = proc (x4, (x3, (x2, (x1, x0)))) -> do
%--      o1 <- aXor -< (x4, x0)
%--      o2 <- aXor -< (x4, x1)
%--      o3 <- aId  -< (x2)
%--      o4 <- aId  -< (x3)
%--      returnA    -< (o4, (o3, (o2, o1)))
%
%crc_polynom_usb ::   Arrow a => Grid a (Bool, (Bool, (Bool, (Bool, (Bool, Bool)))))  (Bool, (Bool, (Bool, (Bool, Bool))))
%crc_polynom_usb
%    =   mvRight >:> mvRight >:>
%        (   aDistr
%        >>> aXor *** (mvRight >:> aXor)
%        )
%
%-- crc_polynom_usb
%--  = proc (x5, (x4, (x3, (x2, (x1, x0))))) -> do
%--      o1 <- aXor -< (x5, x0)
%--      o2 <- aId  -< (x1)
%--      o3 <- aXor -< (x5, x2)
%--      o4 <- aId  -< (x3)
%--      o5 <- aId  -< (x4)
%--      returnA    -< (o5, (o4, (o3, (o2, o1))))
%
%crc_polynom_sdmmc :: Arrow a => Grid a (Bool, (Bool, (Bool, (Bool, (Bool, (Bool, (Bool, Bool)))))))  (Bool, (Bool, (Bool, (Bool, (Bool, (Bool, Bool))))))
%crc_polynom_sdmmc
%    =   mvRight >:> mvRight >:> mvRight >:>
%        (   aDistr
%        >>> aXor *** (mvRight >:> mvRight >:> aXor)
%        )
%
%-- crc_polynom_sdmmc
%--  = proc (x7, (x6, (x5, (x4, (x3, (x2, (x1, x0))))))) -> do
%--      o1 <- aXor -< (x7, x0)
%--      o2 <- aId  -< (x1)
%--      o3 <- aId  -< (x2)
%--      o4 <- aXor -< (x7, x3)
%--      o5 <- aId  -< (x4)
%--      o6 <- aId  -< (x5)
%--      o7 <- aId  -< (x6)
%--      returnA    -< (o7, (o6, (o5, (o4, (o3, (o2, o1))))
%
%    
%
%--                                    3      2      1     0         2      1     0
%inner_crc_3ord :: Arrow a => Grid a (Bool, (Bool, (Bool, Bool)))  (Bool, (Bool, Bool))
%inner_crc_3ord 
%    =   aDistr
%    >>> aSnd *** aDistr
%    >>> second (aXor *** aXor)
%
%
%toInner3
%    =   mvRight 
%    >:> mvRight 
%    >:> aFlip
%
%toInner4
%    =   mvRight
%    >:> mvRight 
%    >:> mvRight 
%    >:> aFlip
%
%toInner5
%    =   mvRight
%    >:> mvRight
%    >:> mvRight
%    >:> mvRight
%    >:> aFlip
%
%toInner7
%    =   mvRight
%    >:> mvRight
%    >:> mvRight
%    >:> mvRight
%    >:> mvRight
%    >:> mvRight
%    >:> aFlip
%
%toInner8
%    =   mvRight
%    >:> mvRight
%    >:> mvRight
%    >:> mvRight
%    >:> mvRight
%    >:> mvRight
%    >:> mvRight
%    >:> aFlip
%
%
%-- crc_checksum_8 crc_polynom polySkip padding start rest
%--     = proc 
%crc_checksum_8 crc_polynom polySkip padding start rest
%    =   (padding &&& aId)
%    >>> toInner8
%
%    >>> (start &&& rest)
%    >>> first (crc_polynom)
%
%    >>> step
%    >>> step
%    >>> step
%    >>> step
%    >>> step
%    >>> step
%
%    >>> aFlip
%    >>> polySkip
%    >>> crc_polynom
%
%    where step =   a_aBC2ABc
%               >>> first 
%                   (   aFlip
%                   >>> polySkip
%                   >>> crc_polynom
%                   )
%
%
%crc_test
%    = crc_checksum_8 
%        inner_crc_3ord
%        toInner3
%        (aConst (False, (False, False)))
%        (second (second (second aFst)))
%        (aSnd >>> aSnd >>> aSnd >>> aSnd)
%
%crc_checksum_ccit_8 ::  Arrow a => Grid a (Bool, (Bool, (Bool, (Bool, (Bool, (Bool, (Bool, Bool))))))) (Bool, (Bool, (Bool, Bool)))
%crc_checksum_ccit_8
%    = crc_checksum_8   
%        crc_polynom_ccit
%        toInner4
%        (aConst (False, (False, (False, False))))
%        (second.second.second.second $ aFst)
%        (aSnd >>> aSnd >>> aSnd >>> aSnd >>> aSnd)
%
%crc_checksum_usb_8 ::   Arrow a => Grid a (Bool, (Bool, (Bool, (Bool, (Bool, (Bool, (Bool, Bool))))))) (Bool, (Bool, (Bool, (Bool, Bool))))
%crc_checksum_usb_8 
%    = crc_checksum_8 
%        crc_polynom_usb 
%        toInner5 
%        (aConst (False, (False, (False, (False, False)))))
%        (second.second.second.second.second $ aFst)
%        (aSnd >>> aSnd >>> aSnd >>> aSnd >>> aSnd >>> aSnd)
%
%crc_checksum_sdmmc_8 :: Arrow a => Grid a (Bool, (Bool, (Bool, (Bool, (Bool, (Bool, (Bool, Bool))))))) (Bool, (Bool, (Bool, (Bool, (Bool, (Bool, Bool))))))
%crc_checksum_sdmmc_8
%    = crc_checksum_8 
%        crc_polynom_sdmmc
%        toInner7
%        (aConst (False, (False, (False, (False, (False, (False, False)))))))
%        (second.second.second.second.second.second.second $ aFst)
%        (aSnd >>> aSnd >>> aSnd >>> aSnd >>> aSnd >>> aSnd >>> aSnd >>> aSnd)
%
%%%%%%%%%%%%%%%%%%%%


Generating Netlists,\ldots

\subsection{Simple Example}

\subsection{CRC}

\subsection{Problems with this approach}

\begin{itemize}
\item Problems with nested tupels. Re-Tupeling is not very handy.
\item Tupels are not an appropriate model for hardware wires. (e.g.\ input for a 3-ary AND: is it (x,(y,z)) or ((x,y),z)) 
\item Type for \hs{arr}-Argument (e.g. \hs{\x y z -> (.\ \.\ \.\ )}) is not available for later calculations (to generate the netlists, usw.)
\end{itemize}

The tupel-problem could be avoided (at least superficially) with the use of Patterson's proc-Notation. The proc-Notation
abstracts the tupels \ldots 


\section{Coping with the Tupel-Problem -- "`Vector-Arrows"'}

\begin{haskell}
class VArrow a where
  arr    :: (Vec n b -> Vec m b) -> a (Vec n b) (Vec m b)
  (>>V)  :: a (Vec n b) -> 
  firsts :: 
  (&&&)  :: 
  (***)  :: 
\end{haskell}

\subsection{Horn-Clause-Solution}

\subsection{Type-Family-Solutions}

-  :+: and :-:
- \ldots

Conclusion: That is dependent typing and Haskell is not (yet) a dependently typed function programming laguage. And all
in all it seems that the Vec-Type together with the Arrows-Instances is too much dependent typing for Haskell

\subsection{Using a dependently types language}

\section{Coping with \hs{arr}-Problem}

\subsection{ForSyDe: Using the AST of the arr argument}

In order to transform a usual Haskell function into an hardware block, ForSyDe stores the AST of that function which is
used later to synthesis VHDL.

\subsection{Show-Type-Class (or Typeable)}

\subsection{Generalized Arrows -- avoiding arr}


\bibliographystyle{plain}
\bibliography{Bibliography}
\end{document}
