%\documentclass[11pt,final,a4paper,leqno]{article}
\documentclass[11pt,final,a4paper]{article}
\usepackage[ngerman,english]{babel}
%\usepackage{ucs}
\usepackage[T1]{fontenc}
\usepackage{color}
\usepackage{graphicx}
\usepackage{subfigure}
\usepackage{listings}
\usepackage{fancyvrb}
\usepackage{textcomp}
\usepackage{setspace}
%\usepackage{MnSymbol}
\usepackage{amsmath} % flush eqations to the left
%\usepackage{amssymb}
%\usepackage{mathtools}

%\usepackage[pdftex]{color}
% Haskell Code
\usepackage{listings}
\usepackage{fancyvrb}
\usepackage{textcomp}
\usepackage{setspace}

\lstnewenvironment{haskell}[1][]{
  \lstset{
    fancyvrb=true, columns=flexible, language=haskell,
    captionpos=b,
    basicstyle=\ttfamily\small\setstretch{1},
    commentstyle=\color{ggray}\slshape,
    upquote=true,
    emph={True,False}, emphstyle=\color{green},
    literate=*{>>>}{{\textcolor{ggray}{>>>}}}{3}%
             {...}{{\textcolor{ggray}{...}}}{3}%
             {\{}{{\textcolor{blue}{\{}}}{1}%
             {\}}{{\textcolor{blue}{\}}}}{1}%
             %{=>}{{$\Rightarrow$ }}{2}%
             {->}{{$\rightarrow$ }}{2},
    stringstyle=\color{red}, showstringspaces=false,
    keywordstyle=\color{blue},#1
}}{}

\newcommand{\hs}[1]{%
  \lstinline[
    fancyvrb=true, columns=flexible, language=haskell,
    captionpos=b,
    basicstyle=\ttfamily\setstretch{1},
    commentstyle=\color{ggray}\slshape,
    upquote=true,
    emph={True,False}, emphstyle=\color{green},
    literate=*{>>>}{{\textcolor{ggray}{>>>}}}{3}%
             {...}{{\textcolor{ggray}{...}}}{3}%
             {\{}{{\textcolor{blue}{\{}}}{1}%
             {\}}{{\textcolor{blue}{\}}}}{1}%
             %{=>}{{$\Rightarrow$ }}{2}%
             {->}{{$\rightarrow$ }}{2},
    stringstyle=\color{red}, showstringspaces=false,
    keywordstyle=\color{blue}]!#1!%
}

\definecolor{gray}{gray}{0.5}
\definecolor{ggray}{gray}{0.7}
\definecolor{green}{rgb}{0,0.5,0}



%\lstnewenvironment{haskell}{%
%  \lstset{}% 
%  \csname lst@SetFirstLabel\endcsname
%}
%{%
%  \csname lst@SaveFirstLabel\endcsname
%}

%\lstset {
%  basicstyle=\small\ttfamily,
%  flexiblecolumns = false,
%  basewidth = {0.5em, 0.45em},
%  literate = {+}{{$+$}}1 {/}{{$/$}}1 {*}{{$*$}}1 {=}{{$=$}}1
%             {>}{{$>$}}1 {<}{{$<$}}1 {\\}{{$\lambda$}}1
%             {\\\\}{{\char`\\\char`\\}}1
%             {->}{{$\rightarrow$}}2 {>=}{{$\geq$}}2 {<-}{{$\leftarrow$}}2
%             {<=}{{$\leq$}}2 {=>}{{$\Rightarrow$}}2 
%             {\ .}{{$\circ$}}2 {\ .\ }{{$\circ$}}2
%             {>>}{{>>}}2 {>>=}{{>>=}}2
%             {|}{{$\mid$}}1
%             {<<<}{{$\lll$}}2 {>>>}{{$\ggg$}}2 {-<}{{$\leftY$}}1
%}
%\lstset{                                % Definition der Sprache, f?r lstListings 
%        language=Haskell,               % Name der Sprache
%        breaklines=true,                % Sollen zu lange Zeilen umgebrochen werden?
%        captionpos=b,                   % Wo soll die Beschriftung stehen? (b - bottom, t - top)
%        tabsize=2,                      % Wie viele Zeichen springt ein Tab?
%        basicstyle=\small,              % Welche Schriftgr??e soll verwendet werden?
%        morekeywords={},    		% W?rter die als Schl?sselw?rter fett gedruckt werden sollen
%        deletekeywords={} 		% W?rter die nicht als Schl?sselw?rter fett gedruckt werden sollen
%}

\lstnewenvironment{haskell}[1][]{
  \lstset{
    fancyvrb=true, columns=flexible, language=haskell,
    captionpos=b,
    basicstyle=\ttfamily\small\setstretch{1},
    commentstyle=\color{ggray}\slshape,
    upquote=true,
    escapechar={@},
    emph={True,False}, emphstyle=\color{green},
    literate=*{...}{{\textcolor{ggray}{...}}}{3}%
             {\{}{{\textcolor{blue}{\{}}}{1}%
             {\}}{{\textcolor{blue}{\}}}}{1}%
             {->}{{$\rightarrow$}}{4},
    stringstyle=\color{red}, showstringspaces=false,
    keywordstyle=\color{blue},#1
}}{}

\newcommand{\hs}[1]{%
  \lstinline[
    fancyvrb=true, columns=flexible, language=haskell,
    captionpos=b,
    basicstyle=\ttfamily\setstretch{1},
    commentstyle=\color{ggray}\slshape,
    upquote=true,
    escapechar={@},
    emph={True,False}, emphstyle=\color{green},
    literate=*{...}{{\textcolor{ggray}{...}}}{3}%
             {\{}{{\textcolor{blue}{\{}}}{1}%
             {\}}{{\textcolor{blue}{\}}}}{1}%
             {->}{{$\rightarrow$ }}{2},
    stringstyle=\color{red}, showstringspaces=false,
    keywordstyle=\color{blue}]!#1!%
}
\definecolor{gray}{gray}{0.5}
\definecolor{ggray}{gray}{0.7}
\definecolor{green}{rgb}{0,0.5,0}


\title{Hardware Design with Functional Datastructures: Functions, Arrows, Tupels, Vectors, and Inspection} 

\definecolor{textgray}{gray}{0.25}
\definecolor{backgray}{gray}{0.95}

\newcommand{\grafik}[4][0.9]{%        % Bild einfügen, [Skalierung], {Dateiname ohne Endung}, {Beschriftung}, {label zum Referenzieren}
    \begin{figure}[ht]%               % before: htbp 
        \begin{center}
            \includegraphics[width=#1\columnwidth]{Images/#2}
            \caption{\label{#4} #3}
            
        \end{center}
    \end{figure}
}

\newcommand{\xfig}[4][0.9] {%      %xfig figure einfügen, [Skalierung], {Dateiname ohne Endung}, {Beschriftung}, {label}
    \begin{figure}[ht]
        \begin{center}
            \graphicspath{{./}{Images/}}
            \scalebox{#1}{%
                \input{Images/#2}
            }
            \caption{\label{#4} #3}
        \end{center}
    \end{figure}
}

\newcommand{\reFLect}{\textit{re\kern-0.07em F\kern-0.07emL\kern-0.29em\raisebox{0.56ex}{ect}}}

\newcommand{\boxit}[1]{\mbox{{\it #1}}}

%\newcommand{\hs}[1]{\mbox{\lstinline[basicstyle=\color{textgray}]!#1!}}
%\newcommand{\hsLine}[1]{$$\mbox{\tt{#1}}$$}


\begin{document}

\section{Introduction}


\section{Hardware Modeling in the Functional Programming Paradigm: Recent Approaches}
\label{recent_approaches}
As stated in the introduction, modeling hardware in functional languages is old news, there have been various approaches in the past. The
beginning dates back into the early 80s where Mary Sheeran came up with a functional HDL muFP\cite{sheeran:muFP}. In the same decade John
O'Donnell presented his functional HDL HDRE\cite{hydra:old,donnell}. These two HDL's gave birth to numerous later approaches that are base
more or less on them. 

One of Sheeran's students, Coen Claessen, came later up with a monadic solution called Lava\cite{claessen:hardware}. Ingo Sanders et al
introduced an approach, ForSyDe \cite{forsyde:phd,forsyde:ieee} that is based on meta programming techniques. Lava, ForSyDe and also Mydra %TODO : Hydra reference in literature
represent HDL's embedded into the functional programming language Haskell; MyHDL \cite{myhdl} is embedded in the functional flavored
interpreting language Python. There is HML\cite{hml}, a simple HDL which is embedded inside ML, another functional programming language. 

Than there are functional languages that are exclusively designed to describe hardware, like the language SAFL (=Statically Allocated
Functional Language)\cite{sharp,sharp:flash,sharp:codesign}. This one compiles direct into a netlists; it does not support dynamic allocated
storage like heaps or stacks since these kind of software features do not map well on silicone. 

Another candidate for the functional metaprogramming languages section is the language \reFLect \cite{reflect}. This one is developed at
Intel and tailored for hardware design and theorem proving. 

There are good reasons why metaprogramming (also called \emph{reflexivity}) is beneficial for the use as HDLs. Metaprogramming allows a
program to have a representation of itself, usually by providing data structure of the abstract syntax tree. With this tool at hand the
developer is able to \emph{compute} parts of their program rather than \emph{write}them. This leads to a direct and natural implementation
of program transformations. Haskell also allows to metaprogramming through a library called ``Template Haskell''\cite{haskell:template}. 

With cryptol\cite{cryptol:programming} another class of functional HDLs enters the scene. While all the other HDL's are general purpose,
cryptol is designed to model cryptographic hardware. A sublanguage of cryptol is designed by Galois, explicit to generate FPGA
implementations \cite{cryptol:fpga}. Cryptol has lately been adopted by an American agency and isn't public developed anymore.

\section{Why Arrows?}
% In this section we give a short introduction to the concept of arrows; we briefly compare them to monads which are better known in the
% functional community. We the show why to use arrows model hardware. 




\section{First Approach: Classical Arrows}
Within this section we present a cycle redundancy check algorithm, implemented with the design method of arrows. The example shows a typical
way to express hardware functional. Also the flexibility of a functional approach is demonstrated via the implementation of multiple CRC's.
How to generate netlists out of the functional data structure is shown later on. After that, the problems with classical arrows are
highlighted and solutions are discussed.


\subsection{CRC - Implementation}
The calculation of a specific CRC to some input data is based upon a method called a polynomial division. The reminder of that devision is
the CRC checksum. The input data added with the CRC checksum, results in $0$ if this is devided by the polynomial.

First of all there are operators that are essential to work with this classical arrows. This is because of their tuple like inner structure.
That means, each circuit processes only a few inputs into outputs. The inputs need to be structured in a way, that the function is only
called with the correct inputs. In example, component $c_0$ produces the outputs $s_0, s_1, s_2, s_3$. Another component $c_1$ processes
$s_0$ and $s_2$; $c_2$ acts on $s_1$ and $s_3$. The process of restructuring the output of one function into an input to another function is
called tuple'ing. A function that does such tuple'ing is \hs{mvRight}. This takes an argument of the structure \((x,(y,z))\) to a different
structure \((y,(x,z))\). 

\begin{haskell}
mvRight :: (Arrow a) => Grid a (my, (b, rst)) (b, (my, rst))
mvRight 
  =   aDistr
  >>> first aSnd
\end{haskell} 

The first arrow that is invoked here is \hs{aDistr} which is the pendant of the distribution law. \hs{aDistr} takes something of structure
\(x, (y, z)\) into something of structure \((x, y), (x, z))\), it distributes the first element over the second one. That arrow is combined
with another one, namely \hs{first aSnd}. In fact, this really is an arrow \hs{aSnd} with a scoop-operator \hs{first} such that \hs{aSnd}
works only on the first element of the tuple. The resulting arrow takes a structure \(((x_1,x_2), (y_1,y_2))\) to a resulting structure
\((x_2, (y_1, y_2))\). In the end the \hs{mvRight} arrow moves the first element of the tuple one step to the right. Of curse this function
assumes a structure where always the first element of a tuple is a value, the second is another tuple such like: \((v_1, (v_2, (v_3, (v_4,
v_5, (\dots)))))\)



Another common task over these list like tuple structure is to apply one arrow to the front and than concentrate with the remaining arrows
on the remaining structure. For this purpose the operator \hs{>:>} helps to clarify the 
\begin{haskell}
  aA >:> aB = aA >>> (second aB)
\end{haskell} 

The operator \hs{>:>} is based on the \hs{>>>} operator with a minor difference. The first supplied arrow is executed over the whole input
data tuple, while the second arrow is only applied to the second part of that tuple. Such an operator comes handy while list like tuple
structures are used and are processed from the left to the right. This is exactly the case for the polynomial division. 



While working with tuple lists, it is often convenient to just reorder the parenthesis around the data so that the selector functions like
\hs{first} result in just the right behavior. Two arrows are mentioned here that exploit the associativity of arrows, namely \hs{a\_aBC2ABc}
and \hs{a\_ABc2aBC}. The somewhat cryptographic names only mirror the changing parenthesis, upper case letters show the two positions that
are enclosed by the brackets, the lower case letter is the isolated position. 
\begin{align*}
  \text{a\_aBC2ABc} &: (a, (b, c)) \rightarrow ((a, b), c) \\
  \text{a\_ABc2aBC} &: ((a, b), c) \rightarrow (a, (b, c))
\end{align*}



The \hs{toInner} function makes use of the previously defined operators. There are several \hs{toInner} functions, that only differ in the
amount of steps the value is pushed into the list like tuple.
\begin{haskell}
toInner5
  =   mvRight
  >:> mvRight
  >:> mvRight
  >:> mvRight
  >:> aFlip
\end{haskell} 



With \hs{crc\_checksum\_8} a higher order function is defined that abstracts the general principle behind an 8 bit CRC sum function.  
\begin{haskell}
crc_checksum_8 crc_polynom polySkip padding start rest
  =   (padding &&& aId)
  >>> toInner8

  >>> (start &&& rest)
  >>> first (crc_polynom)

  >>> step
  >>> step
  >>> step
  >>> step
  >>> step
  >>> step

  >>> aFlip
  >>> polySkip
  >>> crc_polynom

  where step =   a_aBC2ABc
             >>> first 
                 (   aFlip
                 >>> polySkip
                 >>> crc_polynom
                 )   
\end{haskell} 

Before starting to calculate the CRC, the input data need to be properly aligned. Therefore the padding bits, which are all $0$ for the
calculation of the CRC, are ``moved'' to right end of the tuple list which for any 8 bit wide CRC algorithm is done via the \hs{toInner8}.
\[((p_0, p_1), (b_0, (b_1, (b_2, (b_3, b_4))))) \rightarrow (b_0, (b_1, (b_2, (b_3, (b_4, (p_0, p_1))))))\]


Afterwards the arrow actually abstracts the steps to be done to calculate a 8 bit CRC sum. The combined action \hs{(start &&& rest)} splits
the input data stream into the just enough bits that fill the Galois type shift register (the \hs{start} part) and into the remaining bits
(the \hs{rest} part). Keep in mind that the \hs{&&&} operator first duplicates its input and passes each arrow one copy. The starting bits
are then fed into the shift register via the \hs{first (crc\_polynom)} expression. Step by step all the bits are consumed, $1$ bit for every
\hs{step}. The last step is executed without the need of the \hs{first} selector, because there are no reminding bits left and for that
reason the arrow ends with the \hs{aFlip >>> polySkip >>> crc\_polynom} expression. 


\subsection{Simple Example}
There are lots of CRC-Codes out there and every one is equipped with a polynomial that especially fits the needs of its domain. In this
examples the simple CRC-Codes for USB and CCIT are shown. Their polynomials are:
\begin{align*}
  \text{USB} &: x^5 + x^2 + 1 \\
  \text{CCIT}&: x^4 + x   + 1 
\end{align*}

These polynomials describe at which positions the input bit is \hs{xor}'ed with that bit that has traveled all the way through the shift
register. For the USB CRC $\text{bit}_0$ and $\text{bit}_2$ are \hs{xor}'ed with $\text{bit}_5$, for the CCIT CRC the $\text{bit}_0$ and
$\text{bit}_1$ are \hs{xor}'ed with $\text{bit}_4$. 

\begin{figure}
  \centering
  \graphicspath{{./}{Images/}}
  \mbox{%
    \subfigure[USB CRC]{\scalebox{0.5}{\input{./Images/CRC-USB}}} \quad \quad \quad \quad
    \subfigure[CCIT CRC]{\scalebox{0.5}{\input{./Images/CRC-CCIT}}}
  }
\end{figure}



The question is, what \hs{polySkip}, \hs{crc\_polynom}, \hs{padding}, \hs{start} and \hs{rest} values need to be chosen. To summarize this
for the CCIT CRC, the highest exponent is $4$. This means, that:
\begin{itemize}
  \item the \hs{polySkip} needs to be \hs{toInner4}
  \item there are $4$ \hs{padding} bits \hs{aConst (False, (False, (False, False)))} 
  \item the split between \hs{start} and \hs{rest} is after bit $4$
\end{itemize}

So the resulting definition of the CCIT CRC sum is:
\begin{haskell}
crc_checksum_ccit_8 
  :: Arrow a => Grid a 
    (Bool, (Bool, (Bool, (Bool, (Bool, (Bool, (Bool, Bool))))))) 
    (Bool, (Bool, (Bool, Bool)))
crc_checksum_ccit_8
  = crc_checksum_8   
      crc_polynom_ccit
      toInner4
      (aConst (False, (False, (False, False))))
      (second.second.second.second $ aFst)
      (aSnd >>> aSnd >>> aSnd >>> aSnd >>> aSnd)
\end{haskell} 

\subsection{Building Netlists}
The data type of the previous CRC function denotes a \hs{Graph} arrow. This is because the arrow generates a graph which in fact holds all
the information that are needed to generate netlists out of it. Basically a graph could be seen as a different form of a Netlist. In Haskell
a graph is modeled by two lists, one for the nodes and one for the edges.


Every hardware component is identified by an ID that is represented by an integer value. This holds also for the in and output pins of
hardware components. The Pins also have IDs that are identified by an integer value.
\begin{haskell}
type CompID = Int
type PinID  = Int
type Pins   = [PinID]
\end{haskell}


An edge represents a wire between two pins on different components and is identified by the tuple of \hs{CompID} and \hs{PinID} which is
called an \hs{AnchorPoint}. There are also two different anchor points, one that goes into a component called \hs{SinkAnchor} as well as one
that comes from a component, the \hs{SourceAnchor}. 

\begin{haskell}
type AnchorPoint  = (Maybe CompID, PinID)
type SinkAnchor   = AnchorPoint
type SourceAnchor = AnchorPoint
\end{haskell}

Note that the \hs{Maybe} constructor is used here, because there can be edges / wires from and to nowhere. These represent the edges of a
component which are not yet connected to the outside world. This by the way maps also to the real world, where it is normal, that
components which are not yet placed have pins from and to nowhere. 


An edge is a wire, is a connection between two components or to be more precise, it is a connection between two anchor points. 
\begin{haskell}
data Edge 
  = MkEdge { sourceInfo :: SourceAnchor
           , sinkInfo   :: SinkAnchor }
\end{haskell}
The record notation is used, to obtain accessor functions (\hs{sourceInfo} and \hs{sinkInfo}) within one step. 

 
Last but not least the actual graph is defined. It holds the information about the node which is similar to the component. The name is later
used in the VHDL generation to identify the entities. Every node holds a list of sub nodes that represent from what this node is build and
helps to build up complex structures from basic simple building blocks. It also holds a list of the connected edges, as well as the list of
input- and output pins. Accessor function are also gained through the use of the record notation.
\begin{haskell}
data StructGraph
  = MkSG { name    :: String
         , compID  :: CompID
         , nodes   :: [StructGraph]
         , edges   :: [Edge]
         , sinks   :: Pins
         , sources :: Pins }   
\end{haskell}

\subsection{Drawbacks}

This approach basically comes with two problems. 

As the CRC example shows up, all the input data to these hardware components must fit into simple tuples. This is easy to handle if the
components only come with one or two pins and certainly which is the case for simple logic gates. But at the latest if the modeled
hardware component uses more than two pins, different data structures are needed. Up to a certain degree one can simulate a
list-like structure with tuples (as it is done in the above CRC examples). But the simulation comes at a cost. 
A big part of the code is needed for restructuring tuples; the CRC example only is an $8$-bit gate and even more restructuring
will be necessary for modeling real world logic gates. This process becomes more tedious, confusing and error prone, if something
like 32 bit hardware components should be modeled. 

The tedious tuple code could at least be superficially avoided by the use of Patterson's  proc-notation for
arrows\cite{PatersonNewNotation}.  The proc-notation abstracts the tuples code, and the hardware can be described more ``visually'' in the
source code. But the proc-notation is just syntactic sugar and underneath there are still tuples in use. 
The following code shows how the CCIT CRC polynomial can be modeled with the proc-notation: 
\begin{haskell}
crc_polynom_ccit 
 = proc (b4, (b3, (b2, (b1, b0)))) -> do
     b4' <- aId  -< (b3)
     b3' <- aId  -< (b2)
     b2' <- aXor -< (b4, b1)
     b1' <- aXor -< (b4, b0)
     returnA     -< (b4', (b3', (b2', b1')))
\end{haskell} 

\par
But there is also a problem with the notation that comes from the auto generated arrow code. Patterson's notation makes a lot use of the
\hs{arr} function of an arrow. But because it is possible to ``lift'' any function that is passed to \hs{arr} into an arrow, it is
impossible to keep track of the types of that functions. Nevertheless the type of the passed function is needed to generate a \hs{Grid}
arrow, that can be transformed into Netlists later. That is, because there is a direct correspondence between the data type of a function
and their hardware representation (e.g.\ number of in- output pins).

\subsection{Grid-Arrow}
Alongside with the arrow execution, the hardware is generated and stored inside a graph like data structure called a \hs{Grid}.
\begin{haskell}
  newtype Grid a b c = GR (a b c, CircuitDescriptor)
\end{haskell} 
The \hs{Grid} type is a combination of the executable arrow and the descriptive hardware component part. Therefore the two are stored inside
a tuple, where the arrow resides in the first split of the tuple and the hardware graph occupies the second part. A Grid is analog to the
snap circuit toys from childhood. It is a kind of grid surface where hardware elements can easily be combined with each other to compose new
structures and analyze their properties. A \hs{Grid} is not only the surface to combine hardware components, but it also is recursive. So
it could be used as hardware component itself which means a \hs{Grid} is an arrow on its own. 

\subsection{Circuit-Descriptor}
The descriptive part of the \hs{Grid} is the second part of the tuple that comes with the type of a \hs{CircuitDescriptor}. A
\hs{CircuitDescriptor}, like the name suggests, describes a circuit in detail. The combination sequence of the hardware parts, their pin
layouts, their cabling structure and also recursively all subcomponents are represented inside a \hs{CircuitDescriptor}.  A circuit with
such a descriptor is detailed enough to be expressed as a Netlist. 

\par
The actual data type is divided into three\footnote{actually four, because the empty descriptor requires an own one} unique descriptors:
\begin{itemize}
  \item \emph{Combinatorial:} are the parts of a circuit that are describable without a clock or buffered data. For example boolean logic
    gates fall into this category. 
  \item \emph{Register:} A register is used to describe the storage of data on the one hand and is used to clock computations on the other
    hand. The way a register is realized depends on the target platform and there is no need to now it in detail at this point. 
  \item \emph{Loop:} Those Hardware components that are referring them self are handled via the looping section.
\end{itemize}

\begin{haskell}
  data CircuitDescriptor
    = MkCombinatorial
      { nodeDesc :: NodeDescriptor
      , nodes    :: [CircuitDescriptor]
      , edges    :: [Edge]
      , cycles   :: Tick
      , space    :: Area
      }
    | MkRegister
      { nodeDesc :: NodeDescriptor
      , bit      :: Int
      }
    | MkLoop
      { nodeDesc :: NodeDescriptor
      , nodes    :: [CircuitDescriptor]
      , edges    :: [Edge]
      , cycles   :: Tick
      , space    :: Area
      }
    | NoDescriptor
    deriving (Eq)
  type Netlist = CircuitDescriptor
\end{haskell} 

The \hs{MkCombinatorial} and the \hs{MkLoop} parts show the classic functional graph structure where there are two lists. One containing the
edges (\hs{edges :: [Edge]}) and another one with the nodes (\hs{nodes :: [CircuitDescriptor]}).  Additionally there are information that
are the same for every circuit and they are stored inside the (\hs{nodeDesc :: NodeDescriptor}) with the graph. 
\begin{haskell}
  data NodeDescriptor
    = MkNode
      { label   :: String 
      , nodeId  :: ID
      , sinks   :: Pins
      , sources :: Pins
      }
    deriving (Eq)
\end{haskell} 
Additionally some informations about the graphs performance (like space- and time consumption) are stored alongside. 

\par 
Each node itself is a \hs{CircuitDescriptor} and contains a list of \hs{CircuitDescriptor}s, so each node could be described recursively,
which suits hardware very well.

\par 
This data structure contains the typical functional way to describe graphs - two lists - but does not enforce to actually build a graph. It
is a common method to supply \hs{smart constructor}s along with the data type. These constructors are functions that take all the elements
the actual \hs{CircuitDescriptor} needs. Than they check the inputs and only generate an output if the inputs fulfill given constraints. In
that way only valid hardware components can be generated. 

\par
A wire is a connection between two \hs{Pins} of one or two hardware components. A wire connects an output pin - called the source - of one
component to the input pin - called the sink - of anther component. For looping hardware it is possible that the two components are the
same. 
\begin{haskell}
  type Anchor       = (Maybe CompID, PinID)
  type SinkAnchor   = Anchor
  type SourceAnchor = Anchor
\end{haskell}

The \hs{Maybe CompID} of the \hs{Anchor}s definition is due to the in- and output pins of the whole circuit. Most of the hardware components
come with inputs and outputs, which are the pins that the hardware brick come with. These global in- and outputs must be modeled in one way
or the other. Here they have no component id, but only a pin id where they come from or go to. 


\section{Coping with the Tupel-Problem - ``Vector-Arrows''}
So classical arrows are an adequate abstract form to describe hardware, they are modular and the come with a enhanced type system that
helps to reduce errors. The part which does not match the hardware perfect is the way the in- and output pins are structured, the tuples. 

\par
As stated earlier, it is somewhat theoretical to work with hardware that come with $\leq2$ in- or output pins. Tuples prefer $2$ in- or
output kind of hardware. Also tuples allow anything to be an in- or output. In the digital world, there is only hardware that transmits
boolean values. Integers are transmitted by bundling multiple boolean values. Also the developers of Sensors try to digitize pretty early in
their design process. \cite{MakinwaSmartSensor} 

\par
The type \hs{a b c} is too general for a logic gate: the type of the data going into a logic gate should always be the same as the type of
the values coming out of this logic gate. Secondly, a set of wires actually is not a tuple and, for instance, the tuple \hs{(a,(b,(c,d)))}
represents the same set of wires as the tuple \hs{(a,((b,c),d))} and the number of possible (different with respect to syntax and type)
representations of a set of wires exponentially explodes with a growing number wires we are trying to model.

\par
These properties describe hardware to be static in their type of values and dynamic in their amount of values. Arrows are dynamic in the
type of values but are static in their amount of the values. So arrows are exactly opposing the hardwares nature. The question is, can
arrows be transformed into a different kind of arrows? And is this possible in such a way that they match hardware more naturally, without
loosing the other beneficial properties for hardware design.

\par
It is obvious that a vector fulfills exactly the desired hardware behavior. A vector $Vec_n^b$ is a vector of $n$ values of type $b$. So
vectors are static in their data type ($b$) and are dynamic in their value amount ($n$). In the following we sketch some ways to express
these ``Vector-Arrows'' in Haskell and therefore we present a \emph{Horn-Clauses} like approach, an approach with \emph{Type-Families} and
one with \emph{Dependently Typed Arrows}. 

\par
So here is a class definition for a vector arrow. 
\begin{haskell}
class VArrow a where
  arr     :: ((@$Vec_n^{b}$@)-> (@$Vec_m^{b}$@)) -> a (@$Vec_n^{b}$@) (@$Vec_m^{b}$@)
  firsts  :: a (@$Vec_n^{b}$@) (@$Vec_m^{b}$@) -> a (@$Vec_{(n:+k)}^{b}$@) (@$Vec_{(m:+k)}^{b}$@)
  seconds :: a (@$Vec_n^{b}$@) (@$Vec_m^{b}$@) -> a (@$Vec_{(k:+n)}^{b}$@) (@$Vec_{(k:+m)}^{b}$@)
  (>^>)   :: a (@$Vec_n^{b}$@) (@$Vec_m^{b}$@) -> a (@$Vec_m^{b}$@) (@$Vec_k^{b}$@) -> a (@$Vec_n^{b}$@) (@$Vec_k^{b}$@)
  (&^&)   :: a (@$Vec_n^{b}$@) (@$Vec_m^{b}$@) -> a (@$Vec_n^{b}$@) (@$Vec_k^{b}$@) -> a (@$Vec_n^{b}$@) (@$Vec_{(m:+k)}^{b}$@)
  (*^*)   :: a (@$Vec_n^{b}$@) (@$Vec_m^{b}$@) -> a (@$Vec_k^{b}$@) (@$Vec_j^{b}$@) -> a (@$Vec_{(n:+k)}^{b}$@) (@$Vec_{(m:+j)}^{b}$@)
\end{haskell}


\subsection{Horn-like-clause Solution}
A first approach would be to define a list-like data type that is limited in length. Therefore it is necessary to bring a counter, namely
the wire count, into the datatype. Type level natural numbers are the tool to restrict the own list like data type in length. One ends up
with a length limited list, namely a vector. 
% TODO, cite richtig platzieren
\cite{kiselyov:strong_collections}
\begin{haskell}
data Vec n a where
  T    :: Vec 'VZero a -- (@$Vec_0^a$@)
  (:.) :: a -> Vec n a -> Vec ('VSucc n) a -- (@$a \rightarrow Vec_n^a \rightarrow Vec_{n+1}^a$@)
\end{haskell} 

\par
The type level natural numbers are introduced as peano numbers. Peano numbers are expressed trough a starting number, that is usually zero
($VZero$) and a successor function ($VSucc$ $n$) which represents the $n+1$ value. Because Haskell allows the definition of recursive data
types, it is possible to bring natural numbers into a data type. A similar mechanism was showed in 2004 by Kiselyov et al
\cite{kiselyov:strong_collections}. 

\par
The arrow class defines the functions \hs{arr}, \hs{first}, \hs{second}, \hs{(***)} and \hs{(&&&)}. \hs{arr} takes a Haskell function into
an arrow that expects a vector of $n$ $b$'s ($Vec_n^b$) into a new vector of $b$'s of length $m$ ($Vec_m^b$). The sequential concatenation
of arrows is done by the \hs{(>^>)} operator.\hs{first} translates an arrow into another one, that takes it's input from the front element
of a tuple. \hs{second} does the same thing, but with the second element of that tuple. So these two functions are uniquely based upon the
tuple. With vector arrows there are no tuples, and therefore there is no way to differentiate a first value from a second one. 

\par 
What these two functions do is, they split the input data and only show the firsts or the tail to the submitted arrow. This is analogue to
the projection maps of a cartesian product ($\pi : A \times B \rightarrow A$, $\rho : A \times B \rightarrow B$). So the \hs{first}-arrow
only gets the first piece, the \hs{second}-arrow gets the second one. There aren't such projection maps or similar functionality for
vectors. So at this point it's clear that vector arrows can't be real arrows. This finding and its aftermath is issued in
Section~\ref{notRealArrows} in detail. Nevertheless, it is possible to split a vector into two parts, namely the first element and the rest
of the list. Also this can be done from the back end, so that the list is split into a list except for the last element and the last
element. This leaves us with an \hs{init} and a \hs{last} function. The backwards version would be a \hs{head} and a \hs{tail} function.

\par
In any way, Haskell is 

Horn Clauses definitions in the data type

\par
Combined together an arrow class for vector arrows could look like the following:

\begin{haskell}
  class (Category a) => VArrow a where
    arr    :: (VNat n)  
           => (Vec n b -> Vec n b) -> a (Vec n b) (Vec n b)


    init   :: ( VNat n, VNat m
              , VAdd n m nm
              , VEq (VSucc n) nm VTrue
              )   
           => a (Vec n b) (Vec m b) -> (Vec nm b) (Vec nm b) -> a (Vec n b) (Vec n b)

    tail   :: ( VNat n, VNat m
              , VAdd n m nm
              , VEq nm (VSucc m) VTrue
              )   
           => a (Vec nm b) (Vec nm b) -> a (Vec m b) (Vec m b)

    tail'  :: ( VNat m ) 
           => a (Vec (VSucc m) b) (Vec (VSucc m) b) -> a (Vec m b) (Vec m b)

    head   :: ( VNat n, VNat m
              , VAdd n m nm
              , VEq nm (VSucc m) VTrue
              )   
           => a (Vec nm b) (Vec nm b) -> a (Vec n b) (Vec n b)

    last   :: ( VNat n, VNat m
              , VAdd n m nm
              , VEq (VSucc n) nm VTrue
              )   
           => a (Vec nm b) (Vec nm b) -> a (Vec m b) (Vec m b)


    (***)  :: (VNat n, VNat m, VAdd n m nm) 
           => a (Vec n b) (Vec n b) -> a (Vec m b) (Vec m b) -> a (Vec nm b) (Vec nm b)

    (&&&)  :: (VNat n, VAdd n n nn) 
           => a (Vec n b) (Vec n b) -> a (Vec n b) (Vec n b) -> a (Vec n b) (Vec nn b)
\end{haskell}

\par
The classical arrow class definition comes with some default definitions. One of them, \hs{a *** b = first a >>> second b}, defines the
\hs{(***)}-operator with the help of \hs{first} and \hs{second}. The class definition above doesn't allow such usage. This is due to the
changed nature of \hs{first} and \hs{second} which have become \hs{init}, \hs{tail}, \hs{head} and \hs{last}. The problem lies within the
plurality of possibilities to split a list into halves in contrast to the singular possibility to divide a tuple. 

\par
It is possible to divide the list in just the right spot, once an arrow is known. \hs{first f} needs to split the input vector into one that
has exact the length \hs{f} expects as input, and a rest vector. So with \hs{aXor} being an arrow that takes two Booleans to one,
\hs{first} applied to it is an arrow from at least two Booleans to at least one.
\begin{haskell}
  --      a (Vec 2 Bool) (Vec 1 Bool)
  aXor :: a (Vec (VSucc (VSucc VZero)) Bool) (Vec (VSucc VZero) Bool)

  --              a (Vec 2+k Bool) (Vec 1+k Bool)
  first (aXor) :: (VNat k, VAdd (VSucc (VSucc VZero)) k nk, VAdd (VSucc VZero) k mk)
               => a (Vec ((VSucc (VSucc VZero)) + k) Bool) (Vec ((VSucc VZero) + k) Bool
\end{haskell} 
The $2+k$ is the length of the whole input vector and the output is a vector of length $1+k$. A drawback is, that the actual syntax isn't
sugared. That means it is not possible to write just $2+k$ but instead one has to write \hs{VAdd (VSucc (VSucc VZero)) k nk}. This might be
tedious for easy arrows and becomes an error source for larger arrows. 


The instantiation of this class gets really messy. This 

\begin{itemize}
  \item (***) laesst sich nicht durch first a >>> second b erzeugen
  \item compiler will mehr infos (horn clauses)
  \item contexte werden unübersichtlich groß (kompiler sollte dies übernehmen)
\end{itemize}

\par
Because vectors can hold more values than two, and it's likely that these two parts are larger than one, the \hs{first} and \hs{second}
function names are appended with a "`s"'. This gives raise to the two arrow operators \hs{firsts} and \hs{seconds}. 





\begin{itemize}
  \item what is needed to define an arrow with them? 
        \begin{itemize}
          \item arrow is comes with arr, first, second, \dots 
          \item minimal instance definition of an arrow
          \item first \& second have other meaning now
        \end{itemize}

  \item Type Families for Type Level Natural number calculations
  \item Compiler Hints?
  \item Problem with :+:

  \item Arrows are not Real Arrows -> problem? 
\end{itemize}


\subsection{Type-Family Solutions}

\subsection{Dependently-Typed Solution}

\subsection{Vector-Arrows are not Arrows}



-  :+: and :-:
- \ldots

Conclusion: That is dependent typing and Haskell is not (yet) a dependently typed function programming language. And all
in all it seems that the Vec-Type together with the Arrows-Instances is too much dependent typing for Haskell

\subsection{Using a dependently types language}

\section{Copeing with \hs{arr}-Problem}

\subsection{ForSyDe: Using the AST of the arr argument}

In order to transform a usual Haskell function into an hardware block, ForSyDe stores the AST of that function which is
used later to synthesis VHDL.  \subsection{Show-Type-Class (or Typeable)}

\subsection{Generalized Arrows -- avoiding arr}


\bibliographystyle{plain}
\bibliography{Bibliography}
\end{document}
