\documentclass[a4paper,UKenglish]{lipics}
\usepackage{fancyvrb}


\lstnewenvironment{haskell}[1][]{
  \lstset{
    fancyvrb=true, columns=flexible, language=haskell,
    captionpos=b,
    basicstyle=\ttfamily\small,
    commentstyle=\color{ggray}\slshape,
    upquote=true,
    escapechar={@},
    emph={True,False}, emphstyle=\color{green},
    literate=*{...}{{\textcolor{ggray}{...}}}{3}%
             {\{}{{\textcolor{blue}{\{}}}{1}%
             {\}}{{\textcolor{blue}{\}}}}{1}%
             {->}{{$\rightarrow$}}{4},
    stringstyle=\color{red}, showstringspaces=false,
    keywordstyle=\color{blue},#1
}}{}


% Commands
\newcommand{\hs}[1]{%
  \lstinline[
    fancyvrb=true, columns=flexible, language=haskell,
    captionpos=b,
    basicstyle=\ttfamily,
    commentstyle=\color{ggray}\slshape,
    upquote=true,
    escapechar={@},
    emph={True,False}, emphstyle=\color{green},
    literate=*{...}{{\textcolor{ggray}{...}}}{3}%
             {\{}{{\textcolor{blue}{\{}}}{1}%
             {\}}{{\textcolor{blue}{\}}}}{1}%
             {->}{{$\rightarrow$ }}{2},
    stringstyle=\color{red}, showstringspaces=false,
    keywordstyle=\color{blue}]!#1!%
}
\newcommand{\grafik}[4][0.9]{%        % Bild einfügen, [Skalierung], {Dateiname ohne Endung}, {Beschriftung}, {label zum Referenzieren}
    \begin{figure}[ht]%               % before: htbp 
        \begin{center}
            \includegraphics[width=#1\columnwidth]{Images/#2}
            \caption{\label{#4} #3}
            
        \end{center}
    \end{figure}
}
\newcommand{\xfig}[4][0.9] {%      %xfig figure einfügen, [Skalierung], {Dateiname ohne Endung}, {Beschriftung}, {label}
    \begin{figure}[ht]
        \begin{center}
            \graphicspath{{./}{Images/}}
            \scalebox{#1}{%
                \input{Images/#2}
            }
            \caption{\label{#4} #3}
        \end{center}
    \end{figure}
}
\newcommand{\reFLect}{\textit{re\kern-0.07em F\kern-0.07emL\kern-0.29em\raisebox{0.56ex}{ect}}}
\newcommand{\boxit}[1]{\mbox{{\it #1}}}

% Colors
\definecolor{gray}{gray}{0.5}
\definecolor{ggray}{gray}{0.7}
\definecolor{green}{rgb}{0,0.5,0}
\definecolor{textgray}{gray}{0.25}
\definecolor{backgray}{gray}{0.95}


\title{From Arrows to Netlists describing Hardware}

\author[1]{Matthias Brettschneider}
\author[2]{Tobias Häberlein}
\affil[1]{University of Applied Science  \\ Albstadt - Sigmaringen \\ brettschneider@hs-albsig.de}
\affil[2]{University of Applied Science  \\ Albstadt - Sigmaringen \\ haeberlein@hs-albsig.de}

\keywords{functional programming, dependent typed programming, algebraic data structures, arrows, vectors, graphs, netlists}


\begin{document}
\maketitle


\begin{abstract}
This paper describes how to transform a functional DSL into hardware represented by a netlist. In earlier papers we proposed the
usage of an algebraic structure called ``arrows'' (basically an abstraction of Haskell's higher-order type (->)) for describing
DSLs. This structure forms the basis of a novel concept that gives the developer a tool at hand to describe hardware functionally
in a natural way. Aside of that an arrow provides not only a tool to synthesize, but to verify and reason about the input DSL. We
have taken this concept to the next stage, from a static size length arrow into a fixed size length vector arrows fitting much
better to logic gates with a fixed number of in- and output pins. There are many sound possibilities to use the algebraic arrow
data structure to model hardware. This paper presents some of them which showed to be most useful. A simple example, the
implementation of a CRC, is used to illustrate the presented techniques.
\end{abstract}


\section{Introduction}
\label{recent_approaches}
As hardware becomes more and more like software, it is not surprising that there are various approaches to bring software
development concepts into the hardware development process. Until now, the way hardware is defined is mostly influenced by the
imperative/procedural paradigm. Our approach is mainly influenced by the functional programming paradigm.

\par
Modeling hardware in functional languages is not new, there have been various approaches in the past. The beginning dates back to
the early 80s where a group around Mary Sheeran came up with a functional HDL (Hardware Description Language)
muFP\cite{sheeran:muFP}. In the same decade John O'Donnell presented his functional HDL HDRE\cite{hydra:old,donnell}. These two
HDL's gave birth to numerous later approaches that are base more or less on them.

\par
One of Sheeran's students, Coen Claessen, later came up with a monadic solution called Lava\cite{claessen:hardware}. Ingo Sanders
et al introduced an approach, ForSyDe \cite{forsyde:phd,forsyde:ieee} that is based on meta programming techniques. Lava, ForSyDe
and also Hydra \cite{donnell} represent HDL's embedded around the functional programming language Haskell. There is HML\cite{hml},
a simple HDL which is embedded inside ML, another functional programming language.

\par
There are functional languages that are exclusively designed to describe hardware, like the language SAFL (=Statically Allocated
Functional Language)\cite{sharp,sharp:flash,sharp:codesign}. This one compiles direct into a netlist; it does not support dynamic
allocated storage like heaps or stacks since these kind of software features do not map well on silicone.

\par
Another candidate for the functional meta programming languages section is the language \reFLect \cite{reflect}. It's one that is
developed at Intel and tailored for hardware design and theorem proving.

\par
There are good reasons why meta programming beneficial for the use as HDLs. Meta programming allows a program to have a
representation of itself, usually by providing a data structure of the abstract syntax tree. With this tool at hand the developer
is able to \emph{compute} parts of their program rather than \emph{write} them. This leads to a direct and natural implementation
of program transformations.

\par
With cryptol\cite{cryptol:programming} another class of functional HDLs enters the scene. While all the other HDL's are general
purpose, cryptol is designed to model cryptographic hardware. A sub-language of cryptol is designed by Galois, explicit to
generate FPGA implementations \cite{cryptol:fpga}. Cryptol has been adopted by an American agency and isn't public developed
anymore.


\section{Arrows for Hardware development}
Arrows are lately developed concept in functional programming \cite{Hughes98generalisingmonads}. As a generalization of Monads,
Arrows fit the task of hardware development much better. Arrows enable a data flow view onto problems. This fits the hardware
developers view which has always been a data flow view, as in circuit board layouts or in state flow diagrams.

\par
The typical problem one faces while describing hardware through functional programming languages is, a functional function is much
more feature rich than the hardware pendant. In the functional programming world functions are so called ``first class citizens'',
which means that they can be passed around like ordinary values. Therefore another function might not only expect values as
parameters, but also functions. This is similar to C++'s template mechanism, where for example a sorting algorithm is type
independently defined. In the functional paradigm such a behavior is achieved by passing an ordering function to the sorting
mechanism. 

\par
The other effect of functions being first class is called currying. This means that it is perfectly correct for any function to
not return a value but to return a function. Currying is something that is present in every function that is supplied with less
parameters than expected. Such function calls result in yet another function which now expects the not supplied parameters as
arguments. 

\par 
It is clear that these concepts can't be translated into hardware where a function is symbolized by a circuit. A good HDL must
come up with a mechanism to even out these differences. Arrows are just the right method to do the job. They offer an interface to
the feature set of functions. Like in a construction kit an arrow enables the developer to choose from the functional function
features. We therefore are able to define arrows that match exactly the feature set of a hardware circuit.

\par
Another reason for the use of arrow is that a concept which is less feature rich can be modeled more abstract. Typical HDL's do
not take this into account. With the arrow approach we gain a maximum level of abstraction while not permitting features that
can't translate into silicone. Arrows are a good compromise between a shallow embedded approach, where the host language is only
used to construct the netlist and a deep embedded approach, where a data type inside the host language is the netlist. 

\par
The arrow \emph{proc}-notation \cite{PatersonNewNotation} also transports the data flow view into the textual source. It is easy
to generate netlists from arrows, which basically boils down to a pretty print function. This also holds for the generation of
typical HDL code like VHDL. They can be used to simulate behavior or proof properties of the circuit. This section gives a brief
overview of the \hs{Grid}-Arrow that can be used to generate VHDL code or to simulate a circuit. The name Grid is analogously
chosen to the snap circuit toys from childhood, where one also could combine hardware elements arbitrary.

\subsection{Grid-Arrow}
\label{subsect:grid}
The \hs{Grid}-Arrow is a combination of an executable arrow and a descriptive hardware component part. In a first approach one can
imagine the descriptive part to live inside the arrow. But that does not work out, so they are stored equal to each other inside a
tuple. The arrow resides in the first, the graph occupies the second part of the tuple.
\begin{haskell}
  newtype Grid a b c = GR (a b c, CircuitDescriptor)
\end{haskell} 
Alongside with the arrow execution, the generated Netlist is stored inside the graph structure called a
\hs{CircuitDescriptor}. The Grid type is a surface where hardware elements can be combined to compose new structures and analyze
them. It is not only the surface to combine hardware components, it's also recursive. So it is used as hardware component itself
which means a \hs{Grid} is an arrow on its own.


\subsection{Circuit-Descriptor}
The descriptive part of the arrow is called a \hs{CircuitDescriptor}, and describes a circuit in detail. The combination sequence
of the hardware parts, the pin layout, their cabling structure and also recursively the subcomponents are represented inside the
\hs{CircuitDescriptor}. A circuit with such a descriptor alongside has enough details to be expressed as a Netlist.

\par
There are three distinct forms of circuits:
\begin{itemize}
\item \emph{Combinatorial:} are the parts of a circuit that are describable without a clock or buffered data. Boolean logic gates
  or deterministic finite automata are in this category.
\item \emph{Register:} A register is used to describe the storage of data on the one hand and is used to clock computations on the
  other hand. The implementation of a register highly depends on the target platform. There is no need to know the details at
  this point.
\item \emph{Loop:} Hardware components that are referring onto them self are handled in the looping section.
\end{itemize}

\begin{haskell}
  data CircuitDescriptor
    = MkCombinatorial
      { nodeDesc :: NodeDescriptor
      , nodes    :: [CircuitDescriptor]
      , edges    :: [Edge]
      , cycles   :: Tick
      , space    :: Area
      }
    | MkRegister
      { nodeDesc :: NodeDescriptor
      , bit      :: Int
      }
    | MkLoop
      { nodeDesc :: NodeDescriptor
      , nodes    :: [CircuitDescriptor]
      , edges    :: [Edge]
      , cycles   :: Tick
      , space    :: Area
      }
    | NoDescriptor
    deriving (Eq)
  type Netlist = CircuitDescriptor
\end{haskell} 

The \hs{MkCombinatorial} and the \hs{MkLoop} parts reveal a functional graph structure that is build from two lists. One
containing the edges and another one containing nodes. Additionally there are equal information for every circuit. They are stored
inside the \hs{NodeDescriptor} with the graph. At last, information about the hardware's performance (space and
time) is stored alongside with the graph.
\begin{haskell}
  data NodeDescriptor
    = MkNode
      { label   :: String 
      , nodeId  :: ID
      , sinks   :: Pins
      , sources :: Pins
      }
    deriving (Eq)
\end{haskell} 
% Each node itself is a \hs{CircuitDescriptor} and contains a list of \hs{CircuitDescriptor}s, so each node could be described
% recursively, which suits hardware very well.

To avoid corrupted data types, smart constructors are used. They ensure correct formed graphs by checking the incoming data
against given constraints and then generate only valid graphs. For example the combinatorial section only accepts non looping
connections.

\par
A wire is a connection between \hs{Pins} of one or
two hardware components. The output is called the source and the input is called the sink of a component. 


\subsection{Lists through Tuples}
The Haskell arrow interface is build from tuples. Circuits come often with a list like interface. There are operators like in
binary logic that perfectly match the tuple structure, but in most cases the list interface is needed. Circuits ofte come with a
list like interface and therefore we modell this throug tuples. The process of restructuring the output of one function into the
input to another function is called retuple'ing. A function that does such tuple'ing is \hs{aFloatR}. This one takes an argument
of the structure \((x,(y,z))\) to a different structure \((y,(x,z))\).

\begin{haskell}
aFloatR :: (Arrow a) => Grid a (my, (b, rst)) (b, (my, rst))
aFloatR           
  =   aDistr 
  >>> first aSnd
\end{haskell} 

There are two arrows involved here, \hs{aDistr} and \hs{aSnd}. With \hs{aDistr} the first element is distributed over the second
one and with \hs{first aSnd} the first occurrence is dropped. The \hs{aFloatR} arrow moves the first element one position to the
right. Of curse this arrow assumes a structure where the first element of a tuple is a value, the second is another tuple: \((v_1,
(v_2, (v_3, (v_4, v_5, (\dots)))))\)

\par
A common task over the list like tuples is to apply an arrow to the front and than concentrate the remaining calculation on the
inner structure. For this purpose the operator \hs{>:>} helps to clarify the code that describes such computations.
\begin{haskell}
  aA >:> aB = aA >>> (second aB)
\end{haskell} 

The operator \hs{>:>} is a special case of the \hs{>>>} operator. The first supplied arrow is executed over the whole input, while
the second one is applied to only the second part. Such an operator comes handy as long tuple structures are used and are
processed from the left.

\par
While working with tuple lists, it is needed to reorder the parenthesis around the data so that selector functions like \hs{first}
process the right data. This is done by the two arrows \hs{a\_xXX2XXs} and \hs{a\_XXx2xXX} that exploit the associative law of
arrows. The names mirror the changing paranthesis structure, upper case letters show the grouped position.
\begin{align*}
  \text{a\_xXX2XXx} &: (a, (b, c)) \rightarrow ((a, b), c) \\
  \text{a\_XXx2xXX} &: ((a, b), c) \rightarrow (a, (b, c))
\end{align*}



\section{Hardware Specification with (classical) Arrows}
Within this section we present a cycle redundancy check (CRC) \cite{peterson:crc} algorithm, implemented with arrows. Additionally
it's shown how netlists are generated. The section concludes by highlighting some of the problems with this approach.

\subsection{CRC - Implementation}
A CRC - cycle redundancy check - is a method to calculate checksums of data. The calculation of a CRC is based upon a polynomial
division, where the reminder is the CRC. Our approach calculates the CRC with a functional galois type shift register.

\par
The shift register is modeled from a list and the positions of where to feedback. The list is simulated by tuples and therefore
the feedback value is feed back into the list by the \hs{aFloatR} arrow.

\par
Different feedback positions are captured in multiple arrows called \hs{aMoveBy}$_n$. 

\begin{haskell}
aMoveBy5
  =   aFloatR
  >:> aFloatR
  >:> aFloatR
  >:> aFloatR
  >:> aFlip
  where 
    (>:>) aA aB = aA >>> (second aB)
\end{haskell} 

The principle behind every CRC is captured in the arrow \hs{aCRC8} 
\begin{haskell}
aCRC8 polynom polySkip padding start reminder
  =   (padding &&& aId)
  >>> aMoveBy8

  >>> (start &&& reminder)
  >>> first polynom

  >>> step
  >>> step
  >>> step
  >>> step
  >>> step
  >>> step

  >>> aFlip
  >>> polySkip
  >>> polynom

  where step =   a_xXX2XXx
             >>> first 
                 (   aFlip
                 >>> polySkip
                 >>> polynom
                 )   
\end{haskell} 

The CRC input data are alignet and the padding bits are floated to the right end of the tuple list.
\[((p_0, p_1), (b_0, (b_1, (b_2, (b_3, b_4))))) \rightarrow (b_0, (b_1, (b_2, (b_3, (b_4, (p_0, p_1))))))\]

After that, the arrow is an abstract representation of the CRC calculation. By parameterising over \hs{aCRC8}, every
concrete CRC algorithm can be constructed. Due to Haskells inability to dynamicaly change the count of certain expressions in the
function body, our example is limited to 8Bit.


\subsection{Example}
A CRC-Code uses a polynomial that especially fits the needs of the domain. This example shows the CRC-Code for USB. With the
polynomial:
\[
  \text{USB} : x^5 + x^2 + 1 
\]
 
The polynomial describes which positions are feed back and are \hs{xor}'ed with the bit that has traveled through the shift
register. For the USB CRC $\text{bit}_0$ and $\text{bit}_2$ are \hs{xor}'ed with $\text{bit}_5$.

\xfig[.5]{CRC-USB}{USB CRC}{fig:usb-crc}

To summarize this for the USB CRC, the highest exponent is $5$. Therefore:
\begin{itemize}
  \item the \hs{polySkip} needs to be \hs{aMoveBy5}
  \item there are $5$ \hs{padding} bits \hs{aConst (False, (False, (False, (False, False))))} 
  \item the split between \hs{start} and \hs{reminder} is after bit $5$
\end{itemize}



\subsection{Building Netlist-Graphs}
A Netlist is a data structure, that describes how an electronic component is build up. The supporting data structure is a graph
that consists of multiple lists. One list contains all the singleton components where another list holds the connecting wires. 

A wire connects two components, starting from the output pin of one component and ending at the input pin of another
component. Every component is uniquely identified by an integer value. Besides the identifier, every component also has a
descriptive name. This one is used to generate hardware describing code like VHDL which is a little easier to debug later on. The
last essential describing part about the component are the list of in- and outgoing pins. 

All this information are sub summed under the ``NodeDescriptor'' data type. Our NodeDescriptor distinguishes four types of
components. One of them are the combinatorial components. They are defined without any form of clock in mind. It is possible to
compute the result of such a component. Also one could substitute the conglomerate of combinatorial components by a single
function f. 

Multiple combinatorial components are combined to bigger components by a clocked register. The register has an input for a clock
signal and therefore takes care of the clock synchronization. Between every two registers must be at least one combinatorial
component.

A register on its own is separate component that is also listed separate. The buildup of a register is not handled here. The
register is defined per architecture so that a solution is a multi platform solution.

The last type of components are the ones that refer onto themselves. They have a combinatorial core from which some wires point
right back into the same component. So these wires have the same in- and output Id's. Of course such looping components are
synchronized by a register in the loop.


A wire is identified by 4 integer value Id's. Every Pin has one and also every component. Wires that come from the ``outside'' or
that go ``out'' do not have a component there and therefore don't have a sink- and source Id. To also work with such cases, the
component Id's are not only integers but \hs{Maybe Integer}. This has a direct translation into the real world, where components
exist, that aren't built in yet. Exactly this components do have pin's which are connected to the gates but aren't soldered in
yet.

\subsection{Translate the Netlist into VHDL}
With a netlist being an universal description of the hardware circuits it is our goal to convert the hardware describing arrow
into one. Every HDL (hardware description language) comes with quirks that must be respected. To translate a netlist into a
specific hdl like VHDL one considers the idiosyncrasy of VHDL. That are things like the header of a VHDL file, the components
which are described within Entity's or the wires that are called portmaps in VHDL. To the outside, the translation process hides
behind the ``showCircuit'' method.

\par
Functions that name single elements are named starting with ``name'' in the function name. VHDL quirks are represented by the
functions that start with ``vhdl\_''. At last there are the auxiliary functions for the remaining tasks. 

\par
The ``name''-Functions are needed to assign the different elements of the netlist a desired name. The information used to generate
such names are extracted from the component description called the \hs{CircuitDescriptor}. These information are then collected
into a temporary representation called a dictionary. The Id's of the components are bound together with the names of the
components by \hs{zip}ing them. Consider the In- and Outputs: First of all, the \hs{PinID}s are combined with the default-names
for input-wires; the same is done with output pins and output-wire names. This chunk of data is further combined with the
component id's. The result is a component dictionary that contains for every pin a corresponding name.

\par
Specific VHDL quirks are taken care of by the ``vhdl\_'' functions. This functions generate correct VHDL source code by looking at
\hs{CircuitDescriptor}s. Nearly every ``vhdl\_'' function takes one of the dictionary's as input to generate more reasonable
names. Every element that is used in the VHDL context to depict a netlist, is produced e.g. by one of the functions ``Entity'',
``Component'', ``Signal'' or ``Portmap''. A component in the VHDL sense is build from entity's. To generate a VHDL netlist, all
the subnodes from the graph-view are taken. An entity is generated for every one of the subnodes. If all entity's are generated
the VHDL component header is the big bracket around them. 

\begin{haskell}
showCircuit :: CircuitDescriptor ->  String
showCircuit g 
     = concat $ map break
     [ ""
     , vhdl_header
     , vhdl_entity         g namedComps
     , vhdl_architecture g 
     , vhdl_components     g namedComps
     , vhdl_signals        g namedEdges
     , vhdl_portmaps       g namedComps namedEdges
     ]
     where namedEdges = generateNamedEdges g
            namedComps = generateNamedComps g
\end{haskell} % $

\par
To the outside only two functions remain view able. Both of them take something into a \hs{String}, one from a
\hs{CircuitDescriptor} and the other one from an \hs{Edge}.

\par
The execution of the described functions produce a \hs{Grid} arrow as result. As stated in Section \ref{subsect:grid} a Grid
consists of the arrow representing the behavior and the graph representing the structure that causes this behavior. Every
component comes with a trivial graph containing only one node. That node describes the component in detail. 

\par
The hardware components are combined with each other by arrow operators to generate new components with different
behavior. Alongside with the combinations of the hardware a new describing graph is generated. This graph describes now, from what
components the new one is build up. And it also describes how the single parts are interconnected.

\par
To interconnect two circuits into a new one, a function \hs{splice} is used. This function is abstract in the sense that it can
splice two circuits together in different ways. They correspond to the arrow operations. For example graphs that are combined by
\hs{(>>>)} are spliced with the \hs{seqRewire} method. The function connects the sources of circuit 1 with the sinks of circuit 2
one by one. After that step all the remaining sinks of the two circuits are extracted into one list and all the remaining sources
are extracted into another list. These two lists become the new sinks and sources of the generated component. To generate a proper
graph of circuits one has to make sure, that all the identifiers inside the components are again unique. 

\xfig[.5]{combine-components}{combine two components}{fig:combine}

\par
It turns out, that this graph in fact is a form of a netlist. Once the graph is created, it is straight forward to translate it
into another netlist format (e.g. VHDL).

\subsection{Drawbacks}
This approach has problems, that are visible in the CRC example. The input data of these hardware components must fit into
tuples. Logic gates fit well into this scheme, but for hardware components with more than two pins data structures like lists are
needed. Lists can be simulated with tuples as it is done in the above example. But the simulation comes at a cost. A big part of
the code only restructures the tuples. This process becomes tedious, confusing and error prone. At least if bigger real world
hardware components are be modeled. Think about 32 or even 64-Bit ALU's.

\par
The tedious tuple code could be avoided by using Patterson's proc-notation for arrows\cite{PatersonNewNotation}.  The
proc-notation abstracts the tuples to named variables. The hardware can be described more ``visually'' in the source. But this is
just syntactic sugar, there are still tuples in use. The following code shows how the USB CRC is modeled with the proc-notation:
\begin{haskell}
usbPolynom
 = proc (x5, (x4, (x3, (x2, (x1, x0))))) -> do
     o1 <- aXor -< (x5, x0)
     o2 <- aId  -< (x1)
     o3 <- aXor -< (x5, x2)
     o4 <- aId  -< (x3)
     o5 <- aId  -< (x4)
     returnA    -< (o5, (o4, (o3, (o2, o1))))
\end{haskell} 

\par
The second problem affects the resulting auto generated arrow code. Patterson's notation heavily uses the \hs{arr}
function. Because it is possible to ``lift'' any function passed to \hs{arr} into an arrow, it is impossible to keep track of that
outcoming function type. The type information of the passed function is needed to generate a proper component. This makes it
possible to generate component arrows ad-hoc. Nonetheless a proper name to that function is missing in every case.


\section{Coping with the Tupel-Problem - ``Vector-Arrows''}
Arrows match binary hardware pretty good. In this section we show howto overcome the problems with $n$-ary ($n > 2$) hardware
components.

\par
The reason for the arrows preference of binary circuits is it's tuple structure. Also the tuples allow the in- and output to be of
any kind. In the digital world, hardware transmits nearly always Boolean values and also integers are transmitted via bundled
Boolean values. And even the developers of Sensors try to digitize pretty early in their design process. \cite{MakinwaSmartSensor}

\par
So arrows are an adequate abstract form to describe hardware, they are modular and they come with an enhanced type system that
helps to reduce errors. The part which does not match the hardware perfect are the tuples that prefer binary circuits. Therefore
the type \hs{a b c} is far too general for a generic gates: the type of the in going data should match the type of the out coming
values. A set of wires is not a tuple and for instance, the tuple \hs{(a,(b,(c,d)))} represents the same set of wires as the tuple
\hs{(a,((b,c),d))}. The number of possible -- different with respect to syntax and type -- representations of a set of wires
exponentially explodes with a growing number of wires that we are trying to model.

\par
The ideal type for hardware development is the one that is static in their data types and dynamic in the length.  Arrows in
contrast are dynamic in the data type and are static in their amount of the values. So arrows are opposing the hardware's nature
here. It is possible to transform arrows so that they match hardware more naturally, without loosing the beneficial properties for
hardware design.

\par
A vector fulfills the desired data type needs of the hardware behavior. The vector $Vec_n^b$ is one of $n$ values over the type
$b$. The arrow class with vectors instead of tuples is described by the following class definition.

\begin{haskell}
class VArrow a where
  arr     :: ((@$Vec_n^{b}$@)-> (@$Vec_m^{b}$@)) -> a (@$Vec_n^{b}$@) (@$Vec_m^{b}$@)
  firsts  :: a (@$Vec_n^{b}$@) (@$Vec_m^{b}$@) -> a (@$Vec_{(n:+k)}^{b}$@) (@$Vec_{(m:+k)}^{b}$@)
  seconds :: a (@$Vec_n^{b}$@) (@$Vec_m^{b}$@) -> a (@$Vec_{(k:+n)}^{b}$@) (@$Vec_{(k:+m)}^{b}$@)
  (>^>)   :: a (@$Vec_n^{b}$@) (@$Vec_m^{b}$@) -> a (@$Vec_m^{b}$@) (@$Vec_k^{b}$@) -> a (@$Vec_n^{b}$@) (@$Vec_k^{b}$@)
  (&^&)   :: a (@$Vec_n^{b}$@) (@$Vec_m^{b}$@) -> a (@$Vec_n^{b}$@) (@$Vec_k^{b}$@) -> a (@$Vec_n^{b}$@) (@$Vec_{(m:+k)}^{b}$@)
  (*^*)   :: a (@$Vec_n^{b}$@) (@$Vec_m^{b}$@) -> a (@$Vec_k^{b}$@) (@$Vec_j^{b}$@) -> a (@$Vec_{(n:+k)}^{b}$@) (@$Vec_{(m:+j)}^{b}$@)
\end{haskell}

In the following sections we sketch different ways to express such ``Vector-Arrows'' and present different approaches to work with
them.

\subsection{Solution using Horn Clauses}
As vectors are not a purely functional data structure and lists come pretty close to them the first approach would be to constrain
a list data type in length. It is necessary to introduce a counter into the data type. We choose type level natural numbers to
restrict the list in length.
\begin{haskell}
data Vec n a where
  T    :: Vec VZero a
  (:.) :: (VNat n) => a -> (Vec n a) -> (Vec (VSucc n) a)
\end{haskell} 

The type level numbers are introduced as peano numbers. This is similar to the mechanism shown by Kiselyov et al
\cite{kiselyov:strong_collections}.

\par
The \hs{Arrow}-Class of this GADT vector is defined similar to the classical arrow. \hs{firsts} translates an arrow into one that
takes it's the first $n$ inputs from the front of the vector. \hs{second} takes the input in a analog way from the end. The
Problem is to decide, where the ``end'' starts. 

\par
As Haskell is statically typed, everything in the data type is declared. All variables that hold type level numbers must be
defined in the data type context. To divide a vector in a \hs{first} and a \hs{second} half, one must calculate the peano number
of the split position. This is done by type classes like \hs{VAdd}. They must be listed in the type context of the Vector. Working
with this contexts reminds of Prolog's Horn Clauses programming style.

\par
It is possible to divide the list just at the ``right spot'', once the consuming arrow is known. \hs{first f} splits the input
into exact the length that \hs{f} expects as input and a reminder vector. So with \hs{aXor} being an arrow that takes two Booleans
\hs{first} applied to \hs{aXor} must be an arrow from at least two Booleans to at least one.
\begin{haskell}
aXor :: a (Vec (VSucc (VSucc VZero)) Bool) (Vec (VSucc VZero) Bool)

first (aXor) 
    :: (VNat k, VAdd (VSucc (VSucc VZero)) k nk, VAdd (VSucc VZero) k mk)
    => a (Vec ((VSucc (VSucc VZero)) + k) Bool) (Vec ((VSucc VZero) + k) Bool
\end{haskell} 
This unsugared syntax is actually a major drawback. It is not possible to write just $2+k$. Instead one has to add \hs{VAdd (VSucc
  (VSucc VZero)) k nk} as context of the data type. This might only be tedious for easy arrows but it becomes an major error
source for larger ones.

\par
Another drawback with this solution is the way the \hs{VAdd} type class is defined. 


\subsection{Solution using Type Families}
The next approach is to replace the Horn like clauses by actual calculations. Haskell has lately come up with a type family
extension \cite{Yorgey:GHP} that comes in handy for this task.

\par
A type family in a way enables to calculate the type constructor of a data type. A feature the horn-clause solution lacks. The
length of the vector is stored in a peano number. As peano numbers are expressed as a bunch of type constructors, type families
are used to calculate the constructor amount. The Haskell-Wiki describes this vividly: A type family is to a data type, what the
method of a type class is to a function.

\begin{haskell}
type family (n::VNat)   :+ (m::VNat) :: VNat
type instance VZero     :+ m     = m 
type instance VSucc n   :+ m     = VSucc (n :+ m)
\end{haskell}

The first Line defines the type family, with an operator \hs{:+} that operates on peano numbers.  With the addition type family
the arrow could be defined more similar to the original type class.
\begin{haskell}
class Category a => VArrow a where
  arr     :: (Vec n b -> Vec m b)
          -> a (Vec n b)         (Vec m b)

  firsts  :: a (Vec n b)         (Vec m b)
          -> a (Vec (n :+ k) b)  (Vec (m :+ k) b)

  seconds :: ((m :+ k) ~ (k :+ m))
          => a (Vec n b)         (Vec m b)
          -> a (Vec (k :+ n) b)  (Vec (k :+ m) b)

  (***)   :: a (Vec n  b)        (Vec m  b)
          -> a (Vec n' b)        (Vec m' b)
          -> a (Vec (n :+ n') b) (Vec (m :+ m') b)

  (&&&)   :: a (Vec n b)         (Vec m  b)
          -> a (Vec n b)         (Vec m' b)
          -> a (Vec n b)         (Vec (m :+ m') b)
\end{haskell}

\par 
The problem with this approach is the compiler. It is not possible for GHC to deduce arithmetics in the type context. This is also
valid for simpler calculations down to even the equality of $n+m = m+n$. It's possible to supply the compiler with these
information by hand. But this are only hints into a direction. At this point it gets tedious to supply hints for easy deductions
and gets unpractical for complicated arithmetic's.


\subsection{Solution using Dependent Types}
A dependently typed programming language is a language, that mixes expressions with types. It therefore allows the calculation of
types within that same language.  There are multiple languages with dependent types out there. The one with a pretty good
connection to Haskell is Agda\cite{MartinLofIntuitionisticTypeTheory} \cite{Norell07towardsa}. The following describes the arrow
class in Agda with dependent types.


\par
In summary the previous solutions all try to fix a unitary problem. And this problem depends on the fact that vectors aren't
purely functional. At least it's not practicable in Haskell to define and work with them. A vectors length must show up in it's
data type so that a function can match against it. The problem here is to calculate values at the type level. The transfer of the
inner tuples into the vectors changes the inner data structure of arrows. Therefore the projection maps from that data type must
translate to a vector version. \hs{first} and \hs{second} exactly match the tuple projection maps in Haskell, namely \hs{fst} and
\hs{snd}.

\par 
With vectors the \hs{first} statement is straight forward defined. The \hs{second} expression is the head-scratcher. The
implementation of \hs{first} constraints it's inputs to only those, that are equal or larger than the ones \hs{f} expects. The
\hs{second} case adds the right alignment on top. This can be achieved by calculating the size of the sub vector, ignoring him and
applying the function to the reminder. It can also be done via reversing the vector multiple times and than applying the function
to the front of the vector.

\par
While \hs{aSwitch} is of type $Vec_n^{Int}$, the expression \hs{seconds aSwitch} must be of type $Vec_{k+n}^{Int}, k \ge 0$. This
is hard to express in a functional data type. In a dependent language this is easily expressible by the data type.

\par
The arrow type class form Haskell must be translated into a vector arrow data structure inside Agda, where typically records do
the job. Therefore the vector arrow record in Agda is defined like so:

\begin{haskell}
record VArrow (_~>_ : Set -> Set -> Set) : Set1 where 
  field
    arr     : forall {B n m}     
    -> (Vec B n -> Vec B m) -> (Vec B n) ~> (Vec B m)

    firsts  : forall {B n m k}   
    -> Vec B n ~> Vec B m -> Vec B (n + k) ~> Vec B (m+k)    

    seconds : forall {B n m k}   
    -> Vec B n ~> Vec B m -> Vec B (n + k) ~> Vec B (m+k) 

    _>>>_   : forall {B n m k}   
    -> Vec B n ~> Vec B m -> Vec B m ~> Vec B k -> Vec B n ~> Vec B k

    _***_   : forall {B n m k j} 
    -> Vec B n ~> Vec B m -> Vec B k ~> Vec B j -> Vec B (n+k) ~> Vec B (m+j)

    _&&&_   : forall {B n m k}   
    -> Vec B n ~> Vec B m -> Vec B n ~> Vec B k -> Vec B n ~> Vec B (m+k) 

  infixr 2 _>>>_
  infixr 2 _***_
  infixr 2 _&&&_
\end{haskell}

This specifies a record over a function like type \hs{_~>_}. With the \hs{forall} definition one matches the context of a Haskell
type definition. The \hs{Vec} denotes a vector type defined in Agda that is used to specify the arrow record. A close look at the
\hs{firsts} and \hs{seconds} statements reveals, that they have the same type. If \hs{firsts} comes with a type $(n + k)$ then it
would be reasonable to use $(k + n)$ in the type of the \hs{seconds} statement. This can't be done because of the way the addition
is defined. 

\par
With this vector record at hand it is possible to apply the concepts shown earlier. The \hs{Grid}-type is defined in a following
way inside agda.

\begin{haskell}
data Grid (A : Set -> Set -> Set) : Set1 where
  GR : forall {B C} -> (( A B C ) ,  Nat) -> Grid A
\end{haskell}

It is now possible to define the graph constructing function on top of the \hs{Grid}-type. This will also generate a connection
graph of the circuit structure, that can later be translated into all forms of hardware describing code.



\section{Conclusion}
With this paper we showed how to model circuits from arrow descriptions. Arrows bring just the right amount of abstraction into
the hardware design process. This offers on the one hand a precise tool for the developer. On the other hand it stays as flexible
as it is usual in the software design process.

\par
With the help of the data type \hs{Grid} we showed how to include the newly generated arrows alongside with a describing
part. This one is not only stored in a graph structure, but in a hierarchically graph which matches the hierarchically structure
of hardware pretty good. The concept, to construct a circuit from other circuits that also have underlying structure is common in
the hardware design world (e.g. UML-Activity-Diagrams, Petri-Nets, etc.). 

\par
Our concept is mighty enough to be able to generate graphs and therefore netlists from only the arrow definitions. We think of
netlists as a form of graphs. On the other way round the graph is a mathematical description of what a netlist is. To get from a
graph to a netlist only a ``pretty-print'' function is needed.

\par
The classical arrows come with a tuple inner structure which is the foundation of our problems. Although we are able to simulate
list-like behavior it is neither practical to work with nor is it correct in a data type sense. To take arrows from tuples to
arrows with vectors is therefore mandatory.

\par
A proof of the superiority of the functional approach is the modeled CRC circuit. Although we used the classical arrows for
modeling, it wasn't much pain to model the CRC algorithm with them. It is up to the developer to use a decent data layout so that
the tuple structure stays out of the way. The problems that come from the tuple arrows are theoretically gone with a vector
arrow. It is just not possible to define vectors in such a way inside Haskell that blend with the arrows. 

\par
At last we where therefore forced to take the vector arrow concept into a language that can handle them naturally. Also the gained
type safety of the vectors are a benefit the developer can't abdicate while working in a functional language. 

\section{Future Work}
We are going to follow the vector approach inside Agda. The graph generation is something that can be done completely inside a
dependent typed language. And there could be a connection between Agda and Haskell due to the IO Monad, but this connection is
still work in progress. To fully utilize Agda, we have to define the Agda data types much more in detail. Also the combination and
connection algorithms need to be translated from Haskell into Agda.

\par
It might be possible to completely switch over to an Agda solution. But at the moment this is just an idea and needs further
research. Also Agda is a language that is still in development and there are major changes from time to time. So it is not clear
how reasonable a solution completely in agda is. 



\bibliographystyle{plain}
\bibliography{Bibliography}
\end{document}

%  LocalWords:  netlists LocalWords CRC
